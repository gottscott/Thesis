\chapter{Intra-urban temperature variability in Baltimore}
\label{chap:bmore1}
\section{Introduction}
In this Chapter, we reproduce part of an article published in the Journal of Applied Meteorology and Climatology \citep{scott2017intraurban}. This study focuses on Baltimore, Maryland, USA, a mid-sized American city on the Chesapeake Bay. Satellite-derived land surface temperature images from Baltimore show that spatial variability within the city, including the tendency for downtown areas to be warmer than tree-lined areas on the urban periphery (Fig. \ref{heatmap}). Satellite measurements show that the hottest neighborhoods are characterized by little vegetation, few trees, and impervious surfaces; they are also the most underserved economically \citep{Huang20111753}. In such locations, satellite land surface temperatures can exceed those of downtown by 5-10 degrees Celsius. Baltimore experiences hot, humid summers and has a K{\"o}ppen-Geiger climate classification of Cfa, indicating a warm temperate climate, year-round humidity and precipitation, and hot summers \citep{koppengeigerUpdated}. Accordingly, the City of Baltimore considers heat stress management to be a top priority for disaster preparedness and climate change adaptation \citep{baltimoreheatpriority}. The boxed area in Fig. \ref{heatmap} is the target of city interventions to reduce energy use and potentially temper the UHI through urban greening initiatives. Specifically planned are planting additional street trees, creating pocket parks, focusing community outreach and education efforts, and installing so-called cool roofs (roofs made with highly reflective paint or other covering) in the target neighborhoods. While a number of studies circumstantially support the use of urban greening to cool an urban area, most use models or satellite data to fill in the sporadic station availability common in most urban areas \citep{urbangreeningreview}. A number of studies from past decades have examined the effects of urbanization on urban heating as cities have grown, notably in Columbia, Maryland \citep{columbiaUHIlandsberg}, or Phoenix, Arizona \citep{phoenixUHI}, but it is exceedingly rare to have the chance to study urban modifications as they are implemented. Critical to this is a baseline understanding of urban temperature variability at the sub-neighborhood scale before any changes are implemented. 

In this chapter we ask, how much does minimum daily air temperature vary within neighborhoods exhibiting high LST and does this variability effect agreement with the nearest weather station? 
To answer these questions, we present results from a dense, low-cost air temperature sensor network in Baltimore (Fig. \ref{heatmap}) in an unprecedented characterization of air temperature at sub-neighborhood resolution. This network is part of larger interdisciplinary urban heat island project  \citep{earthzinearticle}. Our measurements focus on extensively sampling a thermal hot spot in East Baltimore (see boxed inset of Fig. \ref{heatmap}) using a network of iButton thermometer/hygrometers and weather stations from NOAA-NWS and Davis Instruments. 

\begin{figure}
\noindent\includegraphics[width = \linewidth]{rnw_chapter1/figures/figure01.pdf}
\caption{A thermal land-surface temperature map of Baltimore, Maryland, USA from Landsat 8 \citep{landsat8} on July 2015. The boxed inset surrounds the neighborhoods of focus in this study.}
\label{heatmap}
\end{figure}

\section{Materials and Methods}

\subsection{Observations}
Data used in this paper come from a NOAA-NWS weather station, a Davis Instruments Vantage Pro-2 weather station, satellites, and a network of Maxim Integrated iButtons. The iButtons provide a stand alone thermometer, hygrometer and data logger the size of a standard watch battery and have a reported accuracy of $ .5^{\circ} C $ for the temperature ranges seen during the reporting period. This accuracy was confirmed in laboratory conditions by a random sample of 30 of the buttons (not shown). 
%The iButtons are shielded by a custom radiation shield (Fig. \ref{methods}) that is naturally aspirated and made of White98 F-23, a commercial material manufactured by White Optics that is highly reflective for visible light and most often used in industrial lighting applications. 
The iButton and a radiation shield are attached with plastic zip ties to trees ($90.4\%$), wooden posts  ($2.2\%$), or metal lampposts or street signs ($7.4\%$) and removed at the end of the summer recording period. 

\begin{figure}
\noindent\includegraphics[width = \linewidth]{rnw_chapter1/figures/figure02.pdf}
\caption{Two examples of sensors deployed in trees in an underserved East Baltimore neighborhood characterized by low tree cover, large percent impervious surfaces as well as high vacancy rates. Site a) typifies a park or green space and was listed as green space and partial shade while b) is characteristic of an impervious streetscape and listed as impervious and partial shade. }\label{pictures}
\end{figure}

The iButtons cost relatively little (approximately 70 US dollars for the iButton) for an out-of-the-box measuring and data-logging solution. Together they provide a low-profile micro-weather station that fits in the palm of a hand and may be installed discreetly in neighborhoods with high foot traffic (Fig. \ref{pictures}). Having a discreet sensor allows us to quickly expand the range of monitoring locations beyond our own neighborhoods and social networks. 

We began installing the network of iButtons in June 2015 and left them to record hourly temperature through mid-September (Fig. \ref{fig:mapmean1}). More iButtons were added throughout the summer to refine measurements for a total of 153 iButtons.  One round of data collection occurred during mid-July. By the time of sensor collection in October, 135 remained. Results were checked for different time periods and found to be insensitive of the period of data analysis: changes in the sensor network did not impact our conclusions. Data collection focused on East Baltimore neighborhoods and placed sensors approximately 150 meters apart on five transects, three East/West, and two North/South, ranging from 1.6 to 4 kilometers long. Additional iButtons were installed in neighborhood parks and near weather stations for data validation. Most of the landscapes in this area are homogeneous.  Two or three story brick rowhouses are the main housing stock (Fig. \ref{pictures}a), corresponding to a local climate zone (LCZ) 3 or compact low-rise \citep{lcz}. Much of the landscape variability comes from the presence of trees (LCZ B, scattered trees) or grass and vegetation (LCZ D, low plants), as seen in Fig. \ref{pictures}b, although the center of the transects passes through a four block by four block urbanized zone (LCZ 1, or compact high-rise) that is home to the Johns Hopkins Medical campus. 

\begin{figure}
\noindent\includegraphics[width = \linewidth]{rnw_chapter1/figures/figure03.pdf}
\caption{Locations of the 135 sensors in East Baltimore. Color shows $\overline{T}$, the temporal mean of daily minimum temperatures for June 1-September 15, 2015.}
\label{fig:mapmean1}
\end{figure}

 Each iButton was installed facing north. At each installation location, scientists recorded the landcover as impervious, grass, or soil, the installation site (trees, metal poles, or wood posts), and estimated the amount of shade as full shade, partial shade, or none (that is, full sun). Landcover takes into account the ambient conditions as opposed to the purely local conditions. For example, a tree sitting in a tree well with exposed soil would be listed as impervious if the surrounding area is concrete or asphalt, and a sensor in a vegetated but vacant lot is specified as being in a green space even if the sensor is not located in a park or in an official city green space. Shade measurements, by comparison, are purely local and only take into account the estimated amount of sunlight that the sensors receive. Figure \ref{pictures} illustrate typical sites that were both listed as partial shade; 5.9\% of sensors were in full sun, 45.2\% were in partial shade, and 48.9\% were in full shade. 

As noted in \cite{Chapman:2015aa} and \cite{wmoguide}, urban air temperature sampling comes with unique challenges. Standard meteorological sighting protocols are often inapplicable in urban areas, and it was particularly challenging to find any locations for sensor installation that reflect the average ambient conditions in many of the neighborhoods studied in this paper. For example, trees are thought to be a source of urban cooling \citep{kleerekoper2012}, but our sampling methodology relies on the presence of at least a few trees. A majority of the iButtons were installed on trees ($90.4\%$), but occasionally, no trees were available in what would be considered average residential conditions, causing sampling to shift to locations in vacant and vegetated lots or wooded alleyways. This is perhaps one form of sampling bias. 
Additionally, brick rowhouses dominate the landscape (see Fig. \ref{pictures}) and release heat at night, which may cause street-adjacent readings to differ from those taken farther from buildings. In spite of these challenges, we argue that our sampling method reasonably balances the need for rigorous meteorological standards with the need for data in an under-studied environment.

In addition to the iButton network, daily temperature data is also available from a NOAA-NWS weather station located in downtown Baltimore at the Maryland Science Center, hereafter referred to as the downtown station (Fig. \ref{fig:mapmean1}). We used a 15-year record to calculate an extreme temperature threshold from the $95^{th}$ percentile of minimum daily temperature. As the downtown station is only 15 years old, we checked this threshold against a longer period (1975-2014) at the Baltimore-Washington International Airport NOAA-NWS synoptic station, approximately 12 kilometers SSW of the study site. The two extreme temperature thresholds were found to be consistent with their mean difference of $2^{\circ} C$. NOAA-NWS station thermometers are aspirated and have a reported accuracy of $.56^{\circ} C $ \citep{nwsdata}.

Hourly meteorological data (wind, pressure, relative humidity, and radiation) are also taken from a Davis Instruments Vantage Pro-2 weather station installed at Johns Hopkins University on the roof of Olin Hall (referred to as the JHU Station). A similar station was installed mid-summer in East Baltimore at two meter height in the greened interior of a residential block; the JHU station wind and radiation data was checked against this data and not found to differ significantly. For continuity's sake, only the JHU station data was used. 
The Vantage Pro-2 has a reported accuracy of $ .5^{\circ} C $ for the temperature ranges seen during the reporting period and is naturally aspirated. 

A number of satellite-derived observations are also used. Landsat 8 \citep{landsat8} from 10:46 AM local time on July 16, 2016, the least cloudy image available for the observation period, was used to calculate both LST and albedo. 
To derive LST, band 10 digital number (DN) data is converted to the top of atmosphere at-sensor radiance and then at-satellite brightness temperature following \citep{LSTalgorithm}. To account for the different surface emissivities, we use the USGS land cover map to categorize the brightness temperatures by land type, and assign them an emissivity value as in \citet{alipour2003land}.
Subsequently, we apply a correction for atmospheric water vapor to the brightness temperature following the mono-window algorithm in \citet{doi:10.1080/01431160010006971}. Climatological temperature from the weather station at Johns Hopkins is used to determine the surface air temperature, which is then used to estimate atmospheric temperature aloft.    
 
%To derive LST, an atmospheric correction and land cover-based emissivity correction were applied to band 10 \citep{LSTalgorithm, alipour2003land}. 
Satellite-derived urban-rural differences are largest during the day, whereas the air temperature differences are largest at night, so even though minimum daily $T_{air}$ and daytime satellite temperature measurements do not occur at the same time, they are comparable in the sense that both may be used to assess the UHI intensity. 

The Landsat 8 scene was also used to calculate albedo using a normalized form of \citet{albedoLiang2001} as outlined in \citet{albedoRonSmith}. Satellite-derived tree canopy data at 10-foot resolution (Fig. \ref{fig:mapmean1}) was provided by TreeBaltimore. Elevation data comes from the Maryland Lidar dataset for Baltimore City \citep{marylandelevation} and the park shapefile data was downloaded from Baltimore OpenData.  

\subsection{Measurement Evaluation}
iButtons come pre-calibrated for laboratory settings, but meteorological air temperature must be measured in the shade \citep{wmoguide}. Shielding is then the principal source of error for outdoor temperature measurements, though poor aspiration or sighting thermometers near sources of heat can also contribute to error. The iButtons and shield used in this study were evaluated against a Vantage Pro-2 naturally aspirated weather station and an aspirated NOAA-NWS station. The results, shown in Figure \ref{fig:bias1}, indicate that sensors agree well with station data for minimum daily temperatures. Diurnal results indicate that while sensors agree at night, significant differences are detected during daytime hours ($2^{\circ} C$ or more). As the daytime differences between the iButton and station temperatures were not well correlated with humidity, wind speed, or incoming solar radiation, they are omitted from this analysis. While micro-climate differences may be exaggerated during the day, we argue that a focus on daily minimum temperatures makes sense in the context of the urban heat island, a phenomena that is maximized at night and largely disappears during the daytime. The greatest need for local data is then at night and in the early morning; thus, temperatures are collected hourly for the period of June 1 to September 15 2015 and sub-sampled to obtain $T = T_{min} \left( x_i,t\right)$, a time-series of daily minimum temperatures at each sample location $x_i$. This dataset \citep{mydata} is available for download through the Johns Hopkins Data Archive at https://archive.data.jhu.edu/dvn/.  Minimum daily temperatures occur at approximately 6 am, at which time the mean wind speed is $0.37 m.s^{-1}$ and mode of the wind direction is North-Northeast, although minimum temperatures may not occur simultaneously at each location.
\begin{figure}
\noindent\includegraphics[width = \linewidth]{rnw_chapter1/figures/figure04.pdf}
\caption{
A summary of the difference between selected iButtons and weather stations located a) at JHU and b) downtown. Each column represents the distribution of 68 measurements from one iButton sensor minus the station for daily minimum temperature for July 10-September 15, 2015. Red lines represent the median of the data, the blue box represents the interquartile range ($IQR = \left[Q1, Q3 \right]$), and black lines represent the wide interquartile range, $\left[ Q1-1.5\cdot IQR, Q3+1.5\cdot IQR\right]$. Outliers outside of this range are shown as $+$; for the JHU station evaluation, one outlier from each column ranging from $5-6^{\circ} C$ is omitted. The shaded area is the accuracy of the iButton, $\pm .5 ^{\circ C}$. }
\label{fig:bias1}
\end{figure}
\section{Results}

\begin{figure}
\noindent\includegraphics[width =\linewidth]{rnw_chapter1/figures/figure05.pdf}
\caption{Histogram (a,c) and time series (b,d) of the E. Baltimore temperature data (a,b) and the difference with the downtown station (c,d) for daily minimum temperature for sensors. The station data count in a) is inflated by a factor of 100 to appear on the same scale as the iButton data. Green and orange lines represent sensors located in green spaces and impervious spaces, respectively. An extreme temperature threshold (b, light gray) represents the $95^{th}$ percentile of temperature for the period 2000-2015. Difference with the downtown station in (c) is calculated as $ {\Delta T} =T_i - T_{downtown}$, and spatially averaged in d) to compute $\langle {\Delta T} \left(t\right) \rangle =\langle T_i - T_{downtown} \rangle $ for $i \in \{ \text{impervious}, \text{green space} \}$. Error bars in d) give the spatial standard deviation, representing sensor to sensor variability.
} \label{fig:histogram1}
\end{figure}


%Our measurements indicate that 
The spatial variability of minimum daily air temperatures measured in East Baltimore is smaller than expected. For the summertime (temporal) mean of daily minimum temperatures seen in Fig. \ref{fig:mapmean1}, the standard deviation is $0.9^{\circ} C$, which is small compared to the seasonal range of $20^{\circ}C$ and small even in the context of the range of temporal mean temperatures, $4.15^{\circ} C$. The distribution of all observed air temperatures is nearly normal (Fig. \ref{fig:histogram1}a), with a temporal-spatial mean temperature $\langle  \overline { \Delta T}  \rangle = 21.7^{\circ} C$.  Here, $\langle T \rangle$ refers to spatially averaged temperature, and $\overline{T}$ refers to temporally averaged temperature. This value was calculated by averaging first over time at each sensor and then over space.

The distribution of temperature (Fig. \ref{fig:histogram1}a) measured at the downtown station is similar to the distribution of temperatures in East Baltimore (the station count was inflated by a factor of 100 for easier viewing), despite having 135 times the number of measurements. The mean and range of East Baltimore and the downtown station are also similar, and when East Baltimore and station data are plotted as a time series (Fig. \ref{fig:histogram1}b) very little difference is discernible. The summer of 2015 experienced several periods of extreme heat, seen where the station temperature exceeds the extreme heat threshold, as well as cooler periods (Fig. \ref{fig:histogram1}b); this day-to-day meteorological variability contributes to the wide range of temperatures seen in the histograms and explains why the reported standard deviation of temporal mean temperatures, $0.9^{\circ} C$, is much smaller than the standard deviation of all data, $2.78 ^{\circ} C$. To calculate the following statistics, each measurement in space and time is weighted equally. The distribution of temperature differences with the downtown station, $ {\Delta T} \left(x_i,t\right) =T_i - T_{downtown}$, is nearly normal (Fig. \ref{fig:histogram1}c), and has a low standard deviation ($\sigma = 1.37$). The mean difference $\overline {\Delta T \left(x_i,t\right) } \approx 0.00^{\circ} C$ is negligible, and much less than the precision of either the iButton or downtown station thermometer. As nearly all of the data falls within $\pm 2 ^{\circ} C$, then we conclude that the downtown Baltimore station is a reasonable way to assess average thermal conditions in East Baltimore.

\begin{figure}
\noindent\includegraphics[width = .8\linewidth]{rnw_chapter1/figures/figure06.pdf}
\caption{The relationship between meteorology and the downtown station. Left: relationship between meteorological variables and temperature differences $\overline{ \Delta T }=\langle T_{sensor}- T_{downtown} \rangle$. Right: relationship between meteorological variables and spatial variability within E. Baltimore as shown by $\sigma_{\Delta T} = \sigma \left(T_{sensor}- T_{downtown} \right)$,  the spatial standard deviation of agreement with the downtown station. Meteorological variables come from the JHU station; shown are daily averages lagged by one day. The correlation coefficient $r$ is calculated as the Pearson product-moment correlation coefficient. }
\label{fig:correlation1}
\end{figure}

Mean agreement with the weather station does not appear to vary with weather conditions. This was assessed by correlating meteorological variables with $\langle \Delta T \rangle \left( t \right)=\langle T_{i}- T_{downtown} \rangle$, the spatial mean temperature difference with the downtown station that varies in time (Fig. \ref{fig:correlation1}a,c,e,g). Meteorological variables are calculated as the mean of the previous day (i.e., lagged one day), and the correlation coefficient $r$ is calculated as the Pearson product-moment correlation coefficient. Two-tailed p-values $p$ are computed as well.  
The observation period covered several periods of extreme heat (Fig. \ref{fig:histogram1} b), however, the correlation between the previous day's temperature and spatially averaged daily $\langle \Delta T \rangle$ is insignificant ($r=0.003$, $p=.98$, Fig. \ref{fig:correlation1}a), indicating periods of extreme heat do not affect agreement with the downtown station. 
Increased wind speed, pressure, and radiation all had insignificant correlations with temperature difference (Fig. \ref{fig:correlation1}c,e,g), with $r >.4$). 
The insignificance of these correlation values lead us to conclude that mean sensor agreement is not explained by meteorological conditions. 
 
Additionally, much of the variability in observed air temperature in Fig. \ref{fig:histogram1} is due to temporal variability rather than spatial variability. First, the downtown station and the sensor network have the same standard deviation ($\sigma = 2.8$), despite the fact that the sensor network has 135 times the number of measurements (Fig. \ref{fig:histogram1}a). When the data is time-averaged and this meteorological variability is removed, the standard deviation falls from $2.78 ^{\circ} C$ to $0.9 ^{\circ} C$. 
Second, sensor-to-sensor agreement with the downtown station does not correlate significantly with meteorological variables except for radiation (Fig. \ref{fig:correlation1}b,d,f,h).  This was assessed by correlating meteorological variables with $\sigma \left( \Delta T \right)= \sigma {\left(T\left( x_i \right)- T_{downtown} \right) }$, the time-varying spatial standard deviation of agreement with the station. This measure assesses the sensor-to-sensor variability in station agreement or more broadly, temperature variability in the spatial sense. 
There are insignificant correlations ($ |r|<0.16$, $p >.2$) with mean temperature, wind speed, and pressure. 
The correlation of radiation with temperature variability (Fig. \ref{fig:correlation1}h, $r=0.5$, $ p < 0.000$) is the only significant correlation, and shows that sunnier conditions increase the chance that a given sensor may disagree with the downtown station.

The spatial variability of East Baltimore air temperature is also much less than what is suggested by daytime LST in Fig. \ref{heatmap}. As expected, the mean of minimum daily air temperatures is much lower than that of LST: $\left \langle \overline{T} \right \rangle = 21.1^{\circ} C$ as compared to $ \langle LST \rangle= 43.3^{\circ} C$, respectively. 
The standard deviation $\sigma$ and range $R$ of air temperature are also lower than those of LST:  $\sigma\left(\overline{T}\right) = 0.9^{\circ} C$ as compared to $\sigma\left( LST \right) = 2.07 ^{\circ} C$, and $R\left(\overline{T}\right) = 4.15^{\circ} C$ versus $R\left( LST \right) = 7.9 ^{\circ} C$. 
The discrepancy between air temperature and LST in the mean, variability and range shows the potential caveats of using LST and air temperature interchangeably to diagnose the severity of urban heating. 
 
Much of the sensor-to-sensor spatial variability can then be explained by variability in land cover, and in particular whether a sensor is placed in an area dominated by impervious or green space. Green spaces, or spaces dominated by grass and other vegetation, are cooler than impervious spaces on average by $0.56^{\circ} C$ (Fig. \ref{fig:histogram1}a). While small, the difference is found to be significant by a Welch's t-test (p-value of $1.9\times 10^{-10}$ using the Python library Scipy's \texttt{stats.ttest\_ind} function \citep{scipy}). This difference explains 22\% of the variability in mean minimum daily temperature.  
Green spaces are also on average cooler than the downtown station, whereas impervious spaces are slightly warmer, though this difference is slight: $\langle \Delta T \rangle _{green} = -0.18 ^{\circ} C$ versus $\langle \Delta T \rangle_{imp.} =+0.39 ^{\circ} C$.  As meteorological standards encourage sighting weather stations around fields and vegetated areas \citep{wmo}, it is important to note that micro-climate effects may influence temperature when using standard weather station data to assess local urban conditions even at night. 

\begin{figure} 
\noindent\includegraphics[width =.8 \linewidth]{rnw_chapter1/figures/figure07.pdf}
\caption{Relationship between $\overline{T}$ and surface properties at each sensor located in green spaces (left) and impervious spaces (right) as well as the Pearson product-moment correlation coefficient $r$. }
\label{fig:geospatialcorrelation1}
\end{figure}

Additionally, temperatures correlate significantly with surface properties (Fig. \ref{fig:geospatialcorrelation1}); these correlations are stronger than insignificant correlations with meteorological variables seen in Fig. \ref{fig:correlation1}.
The relationship between surface properties and temperature was assessed by correlating elevation, albedo, a sensor's distance from an official park, and the calculated percent tree canopy cover for a 33 square meter box centered around the sensor with time-averaged $\overline{T}$.
For green spaces, the variables correlating strongest with $\overline{T}$ are tree canopy cover ($r=-0.44$, $p=0.005$), elevation ($r=-0.35$, $p=0.029$), and albedo ($r=0.305$, $p=0.059$).
Notably, green space temperature correlated strongest with tree canopy cover ($r=-0.44$), while for impervious spaces, the correlation is practically zero ($r=-0.03$). This suggests that trees are associated with cooling only inside parks for the percentages of tree canopy cover found in the study area (for most of the locations, this is 0-50\%). 
Park distance exhibited poor correlation for both impervious and green space ($r=0.07$ and $r=0.06$ respectively), though this may be affected by using the official inventory of park locations---the City of Baltimore runs an adopt-a-lot program that allows citizens and community groups to manage vacant lots as parks, gardens, or green spaces without formal recognition, so many of the sensors which were counted in green space were not in official parks, especially in neighborhoods where there are fewer parks. This suggests that the larger parks did not have an impact on temperature outside of their boundaries. 
For impervious spaces, no geographic variable correlated significantly ($p >.75$), indicating that none of these factors were significantly associated with either cooling or warming. 
 Air temperature is also not correlated significantly with LST (Fig. \ref{fig:lst1}), with $p>.75$ for both green and impervious spaces. While LST is not a surface property as it changes over time, it is highly correlated with static surface properties such as distance to park ($r=0.6$). LST and air temperature are both used as measures to diagnose urban heating, so it is notable that on the neighborhood scale the correlation is so poor. 
 
 \begin{figure}
\noindent\includegraphics[width = .8\linewidth]{rnw_chapter1/figures/figure08.pdf}
\caption{Relationship between $\overline{T}$ and LST (Landsat 8) at each sensor located in green spaces (left) and impervious spaces (right) as well as the Pearson product-moment correlation coefficient $r$.} 
\label{fig:lst1}
\end{figure}

Regression analysis confirms that the presence of green space is a predictor of mean air temperature, and the only reliable predictor of the aspects examined (LST, albedo, distance to park, and tree canopy cover). Ordinary least squares regression with elevation and presence of green space (1 for green space, 0 for impervious) against mean daily minimum temperature gives the following result :
\[  \overline{T} = 25.6 - 0.013 x_{elevation} -1.035 x_{greenspace} \]

which explains 24\% of the variability ($r^2=0.242$). 
Elevation is included because it correlated strongly with green space temperature ($r=-0.35$) and because green space occurs at a range of elevations (0-60 m). Both covariates are statistically significant at a 
$ 90\% $
confidence level, but only presence of green space is statistically significant at the 
$ 95\% $ 
confidence level ($p = 0.000$ for green space, $p=0.086$ for elevation). Other tested variables were co-linear, for example, albedo and tree cover, and so could not produce a robust regression result. The regression coefficient for $x_{greenspace} $ indicates that after controlling for elevation, temperatures decrease by $1.035^{\circ} C$ when entering a green space. This is more than the mean difference of $.6^{\circ} C$ between impervious and green space indicated in Fig. \ref{fig:histogram1}a; as green space is more abundant at lower elevations (Fig. \ref{fig:geospatialcorrelation1}a), this suggests that elevation is masking some of the cooling effects of parks and green spaces. 

\section{Discussion}
Though previous studies have shown that parks are around a degree cooler during the day \citep{urbangreeningreview}, it was unexpected that green spaces would be cooler at night. While during the day, latent heat release in parks and vegetated areas is probably the source of cooling, at night, condensation could cause warming. To explain this, we can offer some hyptheses, although an analysis of the mechanisms for urban cooling are beyond the scope of this paper. 
Taking the approach of \cite{oke1991}, the net energy budget of the urban surface can be understood to be: $L+S+R = Q_{net}$  where $L$ represents latent heat flux, $S$ represents sensible heat flux, $R$ the radiative heat flux, and $Q_{net}$ the residual flux, which is non-zero throughout the course of the day. As wind speed during the nighttime hours is low, sensible heat is low, enhanced radiative cooling could explain why that parks are efficient at nighttime cooling. This is further influenced by differing material properties, such as the thermal capacity, conductance, and emissivity. Geometry, and in particular the sky view factor, may also play a role as parks tend to be more open and thus have greater radiative loss.  
While we do not presently see a relationship between park size and temperature, this is limited by the study area and number of parks present in this study, and so we cannot generalize. Perhaps a larger scale study could find such a connection. 

Our conclusions that wind and other meteorological processes seem less important than surface properties is interesting in light of a recent UHI study in Birmingham, UK, where advection was found to play an important role in governing urban heating \citep{birminghamadvection1}. 
While a number of studies in the literature have found that low wind speeds would allow for more temperature heterogeneity within the urban heat island (e.g., \citet{oke82, madisonUHI}), our study found that higher mean daily wind speed correlated with temperature differences with the station, but not temperature variability. Though the correlation values are low ($r=0.17$ for temperature difference and $r=0.04$ for temperature variability), this suggests that any changes associated with increasing wind speed are experienced uniformly within the study area.

As our climate warms, more cities seek cost-effective strategies to cool their neighborhoods, such as greening plans that increase the amount of vegetated surfaces and increase tree canopy (e.g., \citet{kleerekoper2012}). While this paper is not intended to evaluate these strategies, our findings suggest some possible limitations to interventions implemented at the neighborhood scale, especially for the levels of vegetation found in East Baltimore. 
 Our analysis does find that vegetated spaces are significantly cooler than impervious spaces, but this effect is found to be small and very localized: only $1.04 ^{\circ} C$ even after controlling for the effects of elevation. This is not sufficient to offset urban heating, which is often several degrees or more. 
 One possible policy intervention supported by this work is local greening near where residents are likely to congregate, such as planting more and smaller parks or grass sidewalk right-of-ways in residential areas. While tree cover is associated with cooler air temperatures, this only occurs in green spaces and parks. Perhaps there is a synergy between the two; presently, our results would support greening policies that beginning with planting grass or other low vegetation. We caution that the low tree cover canopy amounts present in our study area (0-50\%) may prevent drawing conclusions about the potential of increasing tree canopy, as the overall tree canopy in this neighborhood is low. While these ranges of tree canopy were not found to be sufficient to cool impervious surfaces, there may still be a threshold level above which this isn't true. 
 
We also caution that differing climate zones and city landscapes may prevent the results of this study from being directly applied to other areas. One lesson we think will apply to other cities is how geo-spatial relationships with surface properties may change according to scale; findings that apply at the city-wide scale may not be relevant at the neighborhood and sub-neighborhood level. This has the possibility to complicate possible policy interventions. We found this to be true for relationships between air temperature and LST: we found that the point-to-point correlation between LST and air temperature was poorer than what was indicated in the literature. Such studies looked at this relationship on larger scales, however (e.g., \citet{Kloog2014132, Nichol2012153, Nichol2009276}). This conclusion agrees with \citet{white2013validating}, who found a poor correlation between LST and air temperature when comparing point to point, but had better results when average LST at radii of larger than 200m.
This may also be true for other variables, such as tree canopy cover, or albedo.  



%\begin{figure}
%\noindent\includegraphics[width = .5\linewidth]{figure02.jpg}
%\caption{The sensor and radiation shield.}\label{methods}
%\end{figure}

%\begin{figure}
%\noindent\includegraphics[width = \linewidth]{rnw_chapter1/figures/figure03.pdf}
%\caption{Locations of the 135 sensors in East Baltimore. Color shows $\overline{T}$, the temporal mean of daily minimum temperatures for June 1-September 15, 2015.}
%\label{fig:mapmean1}
%\end{figure}

%\begin{figure}
%\noindent\includegraphics[width = \linewidth]{rnw_chapter1/figures/figure04.pdf}
%\caption{
%A summary of the difference between selected iButtons and weather stations located a) at JHU and b) downtown. Each column represents the distribution of 68 measurements from one iButton sensor minus the station for daily minimum temperature for July 10-September 15, 2015. Red lines represent the median of the data, the blue box represents the interquartile range ($IQR = \left[Q1, Q3 \right]$), and black lines represent the wide interquartile range, $\left[ Q1-1.5\cdot IQR, Q3+1.5\cdot IQR\right]$. Outliers outside of this range are shown as $+$; for the JHU station evaluation, one outlier from each column ranging from $5-6^{\circ} C$ is omitted. The shaded area is the accuracy of the iButton, $\pm .5 ^{\circ C}$. }
%\label{fig:bias1}
%\end{figure}

%\begin{figure}
%\noindent\includegraphics[width =\linewidth]{rnw_chapter1/figures/figure05.pdf}
%\caption{Histogram (a,c) and time series (b,d) of the E. Baltimore temperature data (a,b) and the difference with the downtown station (c,d) for daily minimum temperature for sensors. The station data count in a) is inflated by a factor of 100 to appear on the same scale as the iButton data. Green and orange lines represent sensors located in green spaces and impervious spaces, respectively. An extreme temperature threshold (b, light gray) represents the $95^{th}$ percentile of temperature for the period 2000-2015. Difference with the downtown station in (c) is calculated as $ {\Delta T} =T_i - T_{downtown}$, and spatially averaged in d) to compute $\langle {\Delta T} \left(t\right) \rangle =\langle T_i - T_{downtown} \rangle $ for $i \in \{ \text{impervious}, \text{green space} \}$. Error bars in d) give the spatial standard deviation, representing sensor to sensor variability.
%} \label{fig:histogram1}
%\end{figure}

%\begin{figure}
%\noindent\includegraphics[width = .8\linewidth]{rnw_chapter1/figures/figure06.pdf}
%\caption{The relationship between meteorology and the downtown station. Left: relationship between meteorological variables and temperature differences $\overline{ \Delta T }=\langle T_{sensor}- T_{downtown} \rangle$. Right: relationship between meteorological variables and spatial variability within E. Baltimore as shown by $\sigma_{\Delta T} = \sigma \left(T_{sensor}- T_{downtown} \right)$,  the spatial standard deviation of agreement with the downtown station. Meteorological variables come from the JHU station; shown are daily averages lagged by one day. The correlation coefficient $r$ is calculated as the Pearson product-moment correlation coefficient. }
%\label{fig:correlation1}
%\end{figure}

%\begin{figure} 
%\noindent\includegraphics[width =.8 \linewidth]{rnw_chapter1/figures/figure07.pdf}
%\caption{Relationship between $\overline{T}$ and surface properties at each sensor located in green spaces (left) and impervious spaces (right) as well as the Pearson product-moment correlation coefficient $r$. }
%\label{fig:geospatialcorrelation1}
%\end{figure}

%\begin{figure}
%\noindent\includegraphics[width = .8\linewidth]{rnw_chapter1/figures/figure08.pdf}
%\caption{Relationship between $\overline{T}$ and LST (Landsat 8) at each sensor located in green spaces (left) and impervious spaces (right) as well as the Pearson product-moment correlation coefficient $r$.} 
%\label{fig:lst1}
%\end{figure}

\section{Conclusion}

Summertime measurements in Baltimore, Md. using 135 low-cost air temperature sensors show that much of the spatial variability in daily minimum air temperature is small and that this variability is explained by surface properties, namely, the presence or absence of vegetation, and not well explained by meteorology. The time-averaged minimum daily temperatures have a spatial standard deviation ($0.9$) that is much smaller than the same measure for satellite-derived land surface temperature ($2.07$), and the sensor-measured temperatures agree well with the NOAA-NWS weather station in downtown Baltimore, with an mean difference for all measurements in time and space of $0.00^{\circ}C$. The presence or absence of vegetation affected temperature more than other meteorological and surface properties examined, and time-averaged air temperatures in green spaces are found to be cooler than impervious spaces by about $1^{\circ} C$. Additionally, only temperatures measured inside green spaces correlate significantly with surface properties, in particular tree cover and elevation, whereas temperatures measured over impervious surfaces do not. 

This work suggests that using thermal satellite imagery to estimate the variability of of air temperatures will exaggerate the variability of air temperatures and care must be used when using thermal imagery in place of \textit{in situ} air temperature measurements in order to diagnose urban heating. As the mean differences with the downtown weather station are not statistically significant, these findings support the use of the downtown weather station in Baltimore to assess average thermal conditions even in thermally-identified hot spots. 

This work raises a number of questions. As discussed above, an open quesion is why green spaces are cooler at night. Another question is whether our findings will hold when examined on a city-wide basis.  More work is ongoing to determine this. Hopefully this will help answer questions about how densely temperature must be monitored to capture the sub-neighborhood variability of interest to our partners in public health and urban planning. This study examined areas that were largely homogeneous in terms of the built environment, but for a geographically expanded study, pairing data with a standard measure of urbanization or classification such as a brightness index or the local climate zone classification could help comparisons with ongoing work in other cities.
