\chapter*{Abstract}

The health threat posed by extreme temperature is potentially exacerbated by the urban heat island (UHI), a land-atmosphere interaction which causes temperatures in cities to be elevated by several degrees compared to rural areas. 
In this thesis, we explore urban heating on several spatial and temporal scales to investigate how urban temperature variability and urban-rural thermal differences vary within cities, between cities, and under warmer conditions. 
 Specifically, we first characterize summertime intra-urban temperature variability with the cities of Baltimore, Md., USA, and Nairobi, Kenya. Then, we show how urban-rural differences often diminish under warmer conditions across several temporal and spatial scales. 
 Finally, we turn to a numerical model, and use high resolution urban data to identify important urban heat island characteristics, many of which the model fails to capture.  
 This work underlies the importance of evaluating satellite information and numerical models using local, \textit{in situ} data in order to best contextualize the health and policy ramifications of scientific measurements and we will enhance process-based understanding of urban heating. 
Such an understanding may aid in the development of more targeted interventions to reduce the effects of heat exposure now and in a warmer climate.