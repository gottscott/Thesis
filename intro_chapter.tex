%\documentclass[12pt}{article}
%\begin{document}
\chapter{Introduction}
%\section{Chapter}
\label{chap:intro}
%X thousands of years ago, humans in Mesopotamia began an agricultural revolution that permitted a sedentary lifestyle. People in Ur, X, and X, settled into clusters of homes that were later called the first cities, from the Greek 'civis', referring to a citizen. 
%Thousands of years later, cities continue to draw inhabitants. To accommodate rural residents, cities developed novel forms of architecture and infrastructure. We refer to the human modification of agricultural, pastoral, or natural landscapes into cities as \textit{urbanization}.  
%Currently, most of the world's inhabitants live in cities. This is a trend that is demographers expect to continue throughout the next century as technological progress concentrates economic growth in urban centers. 
%Urbanization affects many facets of the natural landscape, from the water cycle to ecological distribution of species. Urbanization can also affect the atmopshere, and is one of the most significant ways in which humans so do (Kalnay). Alterations to the surface affects surface geometry, surface material properties, and the surface water balance, affecting the thermal, radiative, and X properties. These changes affect the surface atmosphere (the part of the atmosphere located between the land and several kilometers ).  

Extreme temperature is now the deadliest form of climate hazard worldwide \citep{wmo}, and heat waves, extended periods of elevated heat and humidity, present a growing health problem \citep{ipcc}. The impacts of heat waves are potentially exacerbated by growing urban populations as well as the urban heat island (UHI) effect, a land-atmosphere interaction causing cities to be several degrees hotter than rural areas. The effect is most pronounced at night and thus understood to be caused by urban-rural differences in cooling rates \citep{oke82}. Projected economic losses from urban heating have been calculated as high as \$10 trillion USD worldwide by 2100 \citep{estrada2017global}.

Despite its importance to local climate, the UHI is an imperfectly understood phenomenon. Urbanization influences the boundary layer energy balance in many ways, including via surface radiation effects (reduced albedo, increased surface area, altered emissivity), surface energy partitioning (reduced evapotranspiration, changes in ground heat flux), microscale circulations (urban canyon effects, impacts on convection), anthropogenic heat from buildings and transportation, as well as pollution impacts \citep{arnfield2003two, oke1982energetic}. The relevance of each process to the UHI may differ as a function of background climate \citep{zhao2014strong}, season \citep{arnfield2003two}, or time of day \citep{peng2011surface}. This poses a challenge to adaptation, mitigation, and resiliency efforts \citep{stone2012managing}.

Temperature differences between cities and rural surroundings have been measured around the world \citep{oke82}. These measurements have typically been taken assuming an idealized city with a densely populated, highly urbanized core surrounded by more vegetated and sparsely populated suburbs. Correspondingly, temperature is schematized as decreasing with radial distance from the dense urban core.
However, urban areas are neither monolithic nor homogeneous. 
%Different micro-climates cause temperature readings to differ by up to several degrees \cite{ucz}, meaning that a resident's heat exposure may vary by or even within neighborhood. It may also mean that some areas need intervention when other areas do not. 
%Understanding the existence of micro-climates within the urban setting can have important implications for the disaster management and public health spheres by creating opportunities for more targeted interventions to reduce the effects of heat exposure now and in a warmer climate.
Within the literature, variability within the urban heat island is less well quantified than urban-rural differences \citep{arnfield}, even though in some cases has been documented to be as large as the urban-rural difference or enough to offset it \citep{intraurbanvariabilityUHI}. 
That different micro-climates cause temperature readings to differ by up to several degrees \citep{lcz} means that a resident's heat exposure may vary by or even within neighborhood. It may also mean that some areas need intervention when other areas do not. 
A number of recent studies have started closing this gap through installing \textit{in situ} monitoring networks (for example, \citet{birminghamukUHI, madisonUHI,tokyoUHI, minneapolisUHI}) or using private networks (for example,  \citet{vanosstudentthesis} analyses data from NOAA and Earth Networks, Inc.'s UrbaNet). However, these networks are not yet widespread (for a recent list of such networks, refer to \cite{madisonUHI}), and their focus on urban-rural differences smooths over neighborhood and sub-neighborhood level variability which may be of interest to urban planners. 

 Many cities have an urban weather station with data that are available to researchers and decision makers. The term urban describes a wide range of characteristics in the built and natural environment, and so urban stations may be located in areas which differ markedly enough in characteristics from residential neighborhoods to affect their temperature readings by up to several degrees \citep{lcz}. Understanding how temperatures may differ from a centrally monitored temperature is critically important to health professionals and urban planners, who may rely on central stations to assess the health burden of heat, issue local heat-related weather alerts and closures, or target UHI mitigation strategies. Ultimately, understanding the existence of micro-climates within the urban setting can have important implications for the disaster management and public health spheres by creating opportunities for more targeted interventions to reduce the effects of heat exposure now and in a warmer climate.

These health concerns and economic costs make it important to know how the UHI changes over time, and between locations.  Studies have measured the UHI in cities around the world using  in-situ observations,  modeling, and satellite imagery 
(see \cite{arnfield2003two} and references therein). 
One way to assess spatial variability within the urban heat island is through satellite-derived land surface temperature (LST). Some progress has been made developing algorithms to derive air temperature from LST \citep{vancouverLST, sun2005air, Kloog2014132}, but this is not yet widespread and numerous urban heat island studies use LST (e.g., \citet{natureUHIronsmith, Nichol2009276, xu2015monitoring, ho2016comparison, white2013validating}). 
Although LST is not equivalent to the 2-meter air temperature of concern to human comfort, a number studies show a relationship between the surface urban heat island and the urban heat island as measured by air temperature \citep{VoogtOke2003, arnfield, Nichol2012153}, and living in an area with high LSTs has been associated with higher mortality during periods of high heat \citep{lstUHImortality, laaidi2012impact, harlan2013neighborhood, hondula2012fine}. 
%{\bf more recent review than Arnfield 2003 ?}). 
% there isn't one, probably because there really doesn't need to be 
Of particular importance is how the UHI changes during warm periods, and a  number of recent modeling and observational studies have examined this issue. 
Several studies have used numerical models to investigate how the UHI evolves during summertime heatwaves, and concluded that the heatwave amplifies urban-rural temperature differences at night and during the daytime \citep{li2013synergistic,ramamurthy2017impact,li2014effectiveness,li2015contrasting}. While these findings were corroborated by observations at nearby weather stations, these studies all examined only a single event in a single city  (Baltimore for \cite{li2013synergistic}, \cite{li2014effectiveness} and \cite{li2015contrasting}; New York City for \cite{ramamurthy2017impact}), and it is not known if this result applies to other events in other cities.
Further, the conclusions from the observational studies examining this topic are mixed. 
%% Of the observational studies \cite{Zhou2010,gabriel2011urban,schatz2015urban}, many use standard weather station data. 
\cite{gabriel2011urban} reported an enhanced UHI during summertime heatwave nights in Berlin but \cite{zhou2010atlanta} found that this effect in Atlanta depended on which urban station is chosen for analysis. A novel measurement approach using dense networks of low-cost sensors found evidence that the UHI increases during summertime heatwaves in Madison \citep{schatz2015urban}, but findings from Baltimore show that this is not universal \citep{scott2016intra}.  These observational studies examined different periods, different cities, and used different data sets, making comparisons between studies difficult. 

This inconsistency in both methods and results
underscores the need for a systematic, quantitative investigation of the relationship between the UHI and temperature across multiple cities and over different time scales. 
%Here we perform such a study by investigating how the UHI responds to increases in temperature,  over several timescales as well as with space. We adopt a large sample, empirical approach in order to characterize temperature dependency of the UHI. We choose a simple UHI metric, $\Delta T$ between an urban and a rural station that is based on daily minimum or maximum temperature (though we focus primarily on minimum daily temperature)
%and we use exclusively \textit{in situ} meteorological records.   
To address this, this thesis examines temperarature variability within cities and urban rural thermal differences using observations and models. 
%Intemperature variability within the urban core of specific cities, variation in the urban-rural differences across the USA, and in the last chapter, we turn to how a numerical model captures this variability. 
Chapter~\ref{chap:bmore1} looks at case studies of temperature variability in urban Baltimore, Maryland, USA; in Chapter~\ref{chap:nairobi}, we investigate the same in Nairobi, Kenya. In Chapter~\ref{chap:hw}, we examine cities across the US and show how $\Delta T$ is affected by warmer conditions. Finally, in Chapter~ \ref{chap:bmore2} we perform numerical simulations and evaluate these simulations using the \textit{in situ} data. 

This thesis adds to several key gaps in the literature: first, by presenting case a case study of temperature variability within the urban heat island for Baltimore, Md., we improve the typical range of an urban climate study by looking at an unprecedented spatial resolution of temperature measurements. This approach allows the study of several land types, and expands the analysis beyond a typical study that assumes one type of so-called urban and so-called rural site. 

Second, this thesis adds to the literature with its case study of Nairobi, Kenya. Nairobi, located at approximately 1$^\circ$S, typifies cities often neglected in the urban climatology literature: the city is located in the Southern hemisphere, experiences tropical weather, and is in a lower-middle income country. While this study is not the only study on urban heating in Nairobi, it is the first to focus specifically on informal settlement neighborhoods in any nation. Compared to the Baltimore urban heat island, which has been discussed by several papers, the study presented in Chapter~\ref{chap:nairobi} is only the second to be written on urban heating in Nairobi. 

Third, this thesis conducts an analysis of urban heating across the USA. This is the first study nation-wide study examining the urban heat islands, as measured by air temperatures and studies the largest number of cities using a single methodology and dataset for any paper in the literature. The most important contribution, however, is using this large sample of data to examine how urban heating evolves during warmer periods, a question that has an important bearing on how urban micro-climates will evolve as climate change continues. This study is the first to examine this issue for multiple cities. 

To expand the work beyond the range of available observations, models are required. In the final chapter, we present the first study which specifically evaluates a numerical model for the purpose of studying characteristics of the urban heat island. %Finally, the mechanistic analysis presented in this thesis' final chapter adds rigor to the observational studies. 

Together, this body of work highlights the need for continued work to understand the urban heat island, and also highlights the ability of that science to support disaster prevention and urban planning work in order to reduce the effects of heat exposure now and in a warmer climate. 

%\end{document}
