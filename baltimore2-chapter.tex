\chapter{Monitoring and understanding the Urban Heat Island: comparison of observations and models in Baltimore, Maryland}
\label{chap:bmore2}

We have seen in past chapters how dense sensor networks can provide the data necessary to support UHI understanding in some cities. Such networks are not yet widespread, so % and bias from the low-cost sensors used for these networks has been reported \citep{scott2017temperature,terando2017ad}. 
numerical models can play a key role in filling local knowledge gaps by providing a process-based UHI understanding, examining mitigation strategies, and forecasting the potential impact of climate change. 
Urban simulations using the integrated urban Weather, Research, and Forecasting Model (WRF) modeling system \citep{chen2011integrated} have been evaluated using data from cities worldwide in climate zones ranging from desert to sub-tropical, from Phoenix \citep{georgescu2013summer} to Beijing \citep{wang2013modeling} to Sydney \citep{argueso2014temperature} to Rotterdam \citep{theeuwes2014seasonal}. 
These studies evaluate the accuracy of WRF in different ways,
including using observations of temperature (eg, \citet{kusaka2012numerical}), wind, humidity and precipitation \citep{miao2011impacts,chen2011numerical}, pollutant concentrations \citep{brioude2013top}, and energy fluxes at the surface \citep{yang2015enhancing,loridan2012multi}, in the near-surface atmosphere as well as aloft using atmospheric profilers and flight campaigns \citep{li2013development}. 
In the case of surface observations, many studies compare data from the model grid cell located nearest to the observation station (that is, rather than interpolating) and quantify model agreement with observations using statistical metrics such as the mean root mean squared error (RMSE), correlation, bias, hit rate, index of agreement, or coefficient of determination \citep{loridan2010trade,salamanca2011study,lee2011evaluation,chen2011numerical,chen2014wrf,li2013multi}, with RMSEs for 2-meter temperature ranging from $0.6-4.0^\circ$C \citep{kim2013evaluation}. These studies 
all conclude that the WRF-urban modeling system performs well for modeling the urban environment. % by evaluating numerical simulations in a number of ways. 

While some of these studies use an extensive network to evaluate surface characteristics---for example, \cite{miao2011impacts} used a network of 60 stations in China---most studies evaluate the model using only a few stations, potentially affecting conclusions of model accuracy. 
Additionally, few studies evaluate the ability of the model to simulate the problem of interest.
For example, while many studies examine temperature in urban areas, few explicitly examine urban-rural differences or report on urban-rural differences as an evaluation metric, even where the study purpose is to understand urban-rural differences.
This may be problematic because key UHI characteristics such as the timing and intensity of urban-rural differences vary significantly in both observations and models. 
Observations show that generally, mid-latitude, temperate cities experience the largest urban-rural differences at night \citep{oke82,scott2018reduced}, though the timing and intensity of the urban heat island varies by city, climate zone, or even synoptic weather type \citep{hardin2017urban}. 
Similarly, models suggest a larger nighttime UHI than daytime UHI for cities in North America, including Houston \citep{chen2011numerical}, Oklahoma City \citep{hu2013impact}, Mexico City \citep{cui2012seasonal}, Asia, including Tokyo \citep{kusaka2012urban}, Singapore \citep{li2013multi}, Beijing \citep{wang2013contribution}, and Europe, including Rotterdam \citep{theeuwes2014seasonal} and London \citep{grawe2013modelling}.
In other cities, numerical simulations suggest the opposite, namely, that daytime UHI is larger than nighttime. This includes studies of Beijing \citep{zhang2011impact}, Hangzhou City \citep{chen2014wrf}, as well as several studies of Baltimore \citep{zhang2011impact,li2013synergistic, li2015contrasting}; these studies use the same model but with different parameters, possibly affecting results. 
%., a mid-size city on the East Coast of the United States (see Fig. \ref{fig:map}). 

%Using observations and models to understand and model urban climate thus leaves open several questions that could be used to compare studies of different cities
%UHI modeling studies demonstrate that numerical modeling is a salient method to relate temperature to urbanization and improve process-based urban heat island understanding, but leave open several questions that could be used to compare studies of different cities as well as studies using different models or parameterizations.
This Chapter uses observations and models to answer several questions about the urban heat island. 
 First- by how much does air temperature vary within the urban heat island? Second- how does urban heat island intensity vary throughout the day? Previous studies examined how temperature, urban land, and vegetation related in observations and models, but didn't compare the two relationships. This leads to the third question-how well does the model reproduce the observed relationship between land use and temperature? 
To answer these questions,
 we use UHI data taken at an unprecedented spatial and temporal resolution, and then use those answers to focus our model evaluation.    
We do so by returning to Baltimore, Md.%, a mid-size city on the East Coast of the United States (see Fig. \ref{fig:map}) that has been studied using observations \citep{Huang20111753,scott2017intraurban, scott2018reduced,brazel2000tale} and models \citep{zhang2011impact,li2013development,li2013synergistic,li2015contrasting}.  

\begin{figure}[h]
\centering
% when using pdflatex, use pdf file:
% \includegraphics[width=20pc]{figsamp.pdf}
% when using dvips, use .eps file:
% \includegraphics[width=20pc]{figsamp.eps}\
\includegraphics[width = \textwidth]{rnw_chapter4/figures/figure01map.eps}
\caption{a) model domains for the simulations and b) map of land type in Baltimore City (colors) and observation sites (triangles). In (b), each pixel represents one square kilometer.}
\label{fig:map}
 \end{figure}
 
%Baltimore is a heterogeneous city, with a small, central downtown core of buildings greater than three stories and  neighborhoods dominated by two-three story brick rowhomes. Outlying higher-elevation neighborhoods and suburbs are characterized by grass lawns, multi-level detached homes, and trees; tree canopy covered 28 percent of Baltimore in 2015 \cite{grove2011urban}. 
Previous observational studies of Baltimore demonstrate a pronounced urban heat island linked with population growth and urbanization: \cite{brazel2000tale} linked population growth with increasing urban rural temperature differences in Baltimore over decadal timescales and showed that higher land surface temperature was correlated with population density. 
Urbanization has also been linked with changes in the hydrological cycle, in particular,    %linked urbanization with 
increased flash flooding from summer thunderstorms \citep{ntelekos2007climatological}.   
 A later study,
\cite{Huang20111753}, examined the spatial extent of Baltimore's urban heat island in one Baltimore watershed encompassing urban and suburban areas and showed that land surface temperatures varied by 16.58$^\circ$C throughout the study area. Temperature variations were linked with socio-economic factors including high poverty and crime rates and low income and educational attainment. One limitation of this study was the use of land surface temperature; \cite{scott2017intraurban} used an observational network to focus on a land surface temperature local maximum or 'hotspot' and showed that air temperature variability for minimum daily temperature was only $1.0^\circ$C, much lower than suggested by land temperatures. 
A nationwide observational study that included Baltimore \citep{scott2018reduced} examined urban-rural temperature differences using station data and showed that urban-rural temperature differences decreased during warmer conditions, including daily summertime temperature as well as during heatwaves.  Each of these studies was limited by either temporal or geo-spatial extent; our study attempts to add to the literature by identifying key characteristics of air temperature in the urban heat island for both morning and night. 

One advantage of 
modeling studies over observational studies is their ability to focus on non-local phenomena.
In
\cite{zhang2011impact}, WRF-UCM simulations showed that advective heating from upstream urban areas near Washington D.C. was responsible for Baltimore's urban heat excess during one heat event. 
Models can also offer process-based understanding: \cite{li2013modeling} demonstrated how urban land types affect the surface energy balance, with resulting impacts on temperature and precipitation, and \cite{li2013development} increased sub-gridscale variability in land surface characteristics and found significant changes in the nighttime surface energy budget. 
Other modeling studies have focused on heatwave events: \cite{li2013synergistic} studied Baltimore's urban heat island and found that urban-rural temperature differences were greater during a heatwave event, attributed to moisture deficits in urban areas and low wind speeds. 
%For example, \citet{li2013synergistic} examined how Baltimore's urban heat island evolved during a heatwave using numerical simulations, and evaluated the model by comparing a time-series of 2-meter modelled temperature to two observation stations; the authors reported the mean bias and RMSE. 
%Though their analysis shows that the model exaggerates the sign of the diurnal cycle, the authors conclude that in examining urban-rural differences, the biases will cancel.% and thus the model is adequate. % for examining urban-rural thermal contrasts.  
%Other UHI characteristics, such as the spatial variability of temperature, or the relationship between green infrastructure and temperature, were not used to evaluate simulations. 
These modeling studies use a few (<5) observational stations. While they reported evaluation criteria such as the RMSE of observed temperature against observations at a few observation stations, they did not evaluate the model for its ability to simulate the diurnal cycle of the UHI, reproduce spatial variability, or other characteristics identified in observational studies. Thus, while numerical models may add to an understanding of Baltimore's UHI, it is first necessary to see if such models can reproduce key UHI observations.
%Given the gaps in process-based understanding...

In this Chapter, we using a high-resolution dataset to measure both daytime and nighttime temperature variability in Baltimore and quantify several important characteristics of the urban heat island. 
 In Section~\ref{sec:results_obs}, we quantify first the spatial variability air temperature within the urban heat island, second, the relationship between land use or local infrastructure and temperature, and third, the variability of urban heat island intensity throughout the day.
These answers describe key UHI characteristics identified by observations which we use to develop evaluation criteria for numerical simulations in Section~\ref{sec:results_model}. 
By examining how features of the urban landscape, ranging from green infrastructure to the built environment, affect air temperature within the urban environment, we add to the literature by offering an observational network of unprecedented scale---unprecedented both for Baltimore and for observational network used in WRF UHI evaluation---and using that information to improve model evaluation criteria. 



\section{Materials and Methods}\label{sec:methods}%\label{sec:Methods}

\subsection{Observations}
\label{sec:methods_obs}
Temperature observations in this paper come from a 2016 network of iButton thermometer/hygrometers located throughout the Greater Baltimore area (Fig.~\ref{fig:map}b). 
iButtons were installed in 91 sites beginning in May 2016, with the full network installed by July 1, 2016; at the end of the summer, 85 sensors remained. Thus, in this study we examine data taken between July 1, 2016 and August 30, 2016. Nearly all sites (79) were located within Baltimore City; the remaining 6 were north of the city in Baltimore County. During installation, local site data was taken on the surrounding site characteristics: 24 sensors were located in locations dominated by impervious surfaces, 43 were located in locations with grass and low vegetation, 13 sensors with bare or little ground-cover, and 5 sites with a mix of these characteristics. 

 \begin{figure}
\centering
\includegraphics[width=20pc]{rnw_chapter4/figures/figure02.pdf}
\caption{a) schematic of iButton and custom radiation shield, b) an urban observation site, and c) rural site 103.}
\label{fig:ibutton}
 \end{figure}

Thermometers were housed in a custom shield in a custom, naturally aspirated radiation shield made of WhiteOptics White98 Reflector Film and attached to trees and poles as in Fig.~\ref{fig:ibutton}a using a plastic zip-tie. 
Figure~\ref{fig:ibutton}b,c show two installation sites, one in urban East Baltimore, and the other in a rural area, that are typical installations. 
Out of the 85 sensors, 61 sensors were attached to trees, 17 sensors were attached to metal poles, and 7 were attached to wooden posts. Most of these were estimated to be located in partial shade (38) or full shade (24); the remainder were located in full sun (23). The implications of installation details for temperature monitoring are discussed more in \cite{scott2017intraurban}.

%\begin{table}
%\centering
%\begin{tabular}{l l l l c c}
%Landcover & Description &  Modis \# & $\epsilon$ & $\alpha$ & N  \\
%High Intensity & Developed, high intensity & 26& 0.88 &0.1 & 29 \\
%Medium and Low Intensity & Developed, & 25,24& 0.9, 0.88 & 0.1, 0.11& 24\\
%Open Space& Developed, open space &23 & 0.97 & 0.12 & 20\\
%Rural forest&Deciduous forest & 28& 0.93& 0.15& 4\\
%%Urban forest& Urban forest sites &NA & NA & NA & 16\\
%\end{tabular}
%\caption{Land types, corresponding 40 class MODIS number, their description, corresponding emissivity $\epsilon$ and albedo $\alpha$, and number of observation sites corresponding to the land type. Parameters $\epsilon$ and $\alpha$ are taken from the WRF Urban Canopy Model Lookup Table.}
%\label{tab:lcc}
%\end{table}
\begin{table}
\centering
\begin{tabular}{p{2in} p{1in} l l c c}
\toprule
Landcover & Description &  Modis \# & $\epsilon$ & $\alpha$ & N  \\
\midrule
High Intensity & Developed, high intensity & 26& 0.88 &0.1 & 29 \\
Medium \& Low Intensity & Developed, open space & 25,24& 0.9, 0.88 & 0.1, 0.11& 24\\
Open Space& Developed, open space &23 & 0.97 & 0.12 & 20\\
Rural forest&Deciduous forest & 28& 0.93& 0.15& 4\\
\bottomrule
%Urban forest& Urban forest sites &NA & NA & NA & 16\\
\end{tabular}
\caption{Land types, corresponding 40 class MODIS number, their description, corresponding emissivity $\epsilon$ and albedo $\alpha$, and number of observation sites corresponding to the land type. Parameters $\epsilon$ and $\alpha$ are taken from the WRF Urban Canopy Model Lookup Table.}
\label{tab:lcc}
\end{table}


 To describe the land use type (hereafter, land type) of each observation site, we use the 40-class Moderate Resolution Imaging Spectrometer (MODIS) land type included in WRF to categorize sites. Our sites are located in developed high intensity, developed medium and developed low intensity, developed open space, and non-developped deciduous forest land.
 Table \ref{tab:lcc} shows each land type examined in this study and the corresponding MODIS classifications, which are the same with the exception that we combine the developed, medium intensity classification (MODIS category 25) with the developed, low intensity classification (MODIS category 24) to make a combined medium and low intensity category. Our local site data shows that all developed land types can include both impervious surfaces but also grassy areas. 
 %We also add an urban forest classification, not described by MODIS but available from our own \textit{in situ} site description. 
 % also add something about street trees? 
For developed land types, this study includes 20-30 sites for each category; only 4 sites are included in the rural/forest area. 

These site classifications correspond to a range of possible vegetative fractions. In order to calculate vegetation fractions quantitatively, we calculate satellite observations of Normalized Digital Vegetation Index (NDVI), the normalized difference of near-infrared light $NIR$ to visible red light $VIS$: 
\[NDVI = \frac{NIR-VIS}{NIR+VIS}\]
at each observation site. The data is downloaded from Landsat-8 using the Google Earth Engine platform. We selected the least cloudy scene during the period June 1, 2016- August 15, 2016.  
We calculate the distance to urban parks using park data downloaded from Open Baltimore (https://data.baltimorecity.gov/Culture-Arts/Parks/3r8a-uawz) and use Maryland LiDAR elevation data from the Maryland GIS Data Catalogue 
(http://data.imap.maryland.gov/datasets/e40871e90800420ea90b0fcdf9a36063). 

\begin{figure}
\centering
\includegraphics[width = \textwidth]{rnw_chapter4/figures/ibuttonbias.eps}
\caption{A comparison of iButton (blue solid line) and ASOS (grey dashed line) temperature observations: a) distribution of temperatures for all hours,  6am, and 4pm, b) differences between iButton and ASOS instruments for each category, and c) average summertime temperature by hour. In a and b), the solid black line indicates the mean, the box surrounds the first through third quartiles, and whiskers delineate the wide interquartile range, 1.5 times the first through third quartiles. Data points falling outside this range are marked as 'x'. }
\label{fig:bias}
\end{figure}

iButtons have been evaluated in the literature and have been reported to agree well particularly for nighttime temperatures, but exhibit a warm bias \citep{scott2017temperature,terando2017ad}. Thus, we co-locate one of the iButtons with an Automated Surface Observation System (ASOS) station in downtown Baltimore (station DMH, downloaded from http://www.nws.noaa.gov/asos/). This station provides sub-hourly measurements with reported accuracies of $< 0.5^\circ $C. In Fig.~\ref{fig:bias}, we compare the ASOS measurements with the iButton measurement. We find that the measurements agree well--the mean difference (iButtons-ASOS) is $0.44^\circ$C, the distributions of summertime temperatures for all hours (Fig.~\ref{fig:bias}a) is similar, and the correlation for the two instruments is $r = 0.96$. 

However, hourly measurements agree less well: the 6am error is $-0.29^\circ$C, and the 4pm error is $1.18^\circ$C. The average summertime hourly temperature, Fig.~\ref{fig:bias}c), shows that in addition to a overestimate of afternoon temperature, the diurnal temperature cycle differs between the instruments. Whereas the iButton indicates that $T_{min}$ occurs at 6am, the ASOS instrument indicates 5am. This is also born out in the fact that the mean error for $T_{min}$ is lower than for 6am data, though the same is not true for $T_{max}$ and 4pm data. In order to compensate for this instrument bias, we subtract the hourly difference between the iButton and the ASOS station as shown in Fig.~\ref{fig:bias}c from each iButton instrument on each day. Thus, the iButton measurements reported in later parts of this paper represent a corrected temperature.  

\subsection{Numerical Model} 
\label{sec:methods_model}
In this Chapter, we use the WRF model, a numerical weather model that incorporates the fully
compressible, nonhydrostatic Euler equations \cite{skamarock2008description}. %(Skamarock et al. 2008). 
For sub-gridscale processes that cannot be resolved by the model, a large number of physics schemes are 
used for parameterization of the processes, chosen from among the
several different physics combinations suggested in
previous studies such as \cite{argueso2011evaluation, giannakopoulou2012persian, efstathiou2013sensitivity, zittis2014comparison}. %Argüeso et al. 2011; Giannakopoulou and Toumi 2012; Efstathiou et al. 2013; Zittis et al. 2014).
In this paper, we use the
Community Atmosphere Model (CAM), version 5.1 \citep{neale2010description} %(Neale et al. 2010), 
for microphysics; CAM, version 3 \citep{collins2004description} %(Collins et al. 2004), 
for both longwave and shortwave radiation;
the Noah land surface model \citep{ek2003implementation} %(Ek et al. 2003) 
for land surface processes; 
the Yonsei University scheme \citep{hong2006new} %(Hong et al. 2006) 
for planetary boundary layer processes;
and the Kain–Fritsch cumulus scheme \citep{kain1990one}. %(Kain and Fritsch 1990).

%The initial and boundary conditions
%used for the model runs are from the ...
%reanalysis daily averages at ...resolution. We update the boundary conditions with SST data from the X data source. 
The initial and boundary conditions used for the model runs are from the NCEP North American Regional Reanalysis (NARR; \cite{mesinger2006north}) daily averages 3-hourly (8 times daily) data at approximately 0.3 degrees (32km) resolution. 
We update the boundary conditions with SST data from the NCEP real-time global SST Analysis data source (RTG{\_}SST{\_}HR; \cite{thiebaux2003new}). 
We run the WRF module for 4 months over three nested domains, each with 40 vertical levels: the first domain is run at a resolution of 9km (-85$^\circ$ W to -68$^\circ$W, 32$^\circ$N to 44$^\circ$N ), the second domain is run at a resolution of 3km (-78$^\circ$W to -73$^\circ$W, 36$^\circ$N to 41$^\circ$N), and the third domain is run at a resolution of 1km (-77$^\circ$W to-75$^\circ$W, 38$^\circ$N to 40$^\circ$N). This last 1km domain is run using the urban canopy model in \cite{chen2011integrated} with the cumulus parameterization turned on. 
As our analysis focuses on Baltimore, this paper uses the model output from a 40 square kilometer (km) domain around Baltimore.
% Note: fill in here!!!! 

To simulate urban areas, we make use of the integrated WRF urban modeling system \citep{chen2011integrated}, which couples an urban canopy model (UCM) with the Noah land surface model to compute the surface fluxes. In our study, the Building Energy Model and Building Energy Paramaterization described in \cite{chen2011integrated} are turned off. 
At each urban grid cell, the model computes fluxes from anthropogenic surfaces, which are combined from fluxes from natural surfaces by the land surface model. 
For example, total sensible heat for a given grid cell is calculated as \[ H = F_{veg} \times H_{veg} + F_{urban}\times H_{urban}\]
where $F_{veg}$ is the fractional coverage of natural surfaces, $H_{veg}$ is the sensible heat flux over natural surfaces, $F_{urban}$ is the fractional coverage of urban anthropogenic surfaces, which includes roads, buildings, and roofs, and $H_{urban}$ is the sensible heat flux emanating from these anthropogenic surfaces. Similarly, latent heat is a function of both anthropogenic and natural surfaces' contributions to latent heating: 
\[LH = F_{veg} \times LH_{veg} + F_{urban}\times LH_{urban}\]
where $LH_{veg}$ is the latent heat flux over natural surfaces and $LH_{urban}$ is the latent heat flux emanating from these anthropogenic surfaces.
The WRF urban modeling system has the capability to simulate urban irrigation, green roofs, and anthropogenic heat; in our simulation, these processes are all turned off, so anthropogenic moisture inputs are negligible.

Each urban flux $F_{urban}$ is modeled as the total contributions from each anthropogenic surface, which includes roof, wall, and road: 
\[ F_{urban} = R\cdot F_{roof} + W\cdot F_{building} + RW\cdot F_{road} \]
where $R$ denotes the roof area fraction, $W$ denotes the building wall area fraction, $RW$ denotes the road area fraction. 
For sensible heat, the fluxes $H_i$ for $i=$building, rood, roof, is modeled as: 
\[H_i = \alpha_i\cdot U_c/U_a \cdot \left( T_i - T_{i_0}\right) / \left( \rho c_p \right ) \] 
where $\alpha_i $ is the albedo, $U_c$ is the canopy wind, $U_a$ is the wind in the first atmospheric layer, $T_i$ is the temperature of the first atmospheric layer, $T_{i_0}$ is the temperature of the surface, $rho$ is density and $c_p$ is the atmospheric specific heat). In the case of roads, $U_c = U_a$.
In the case of latent heating, the fluxes are calculated as: 
\[LH_i = \alpha_i \cdot U_c/U_a \cdot BET_i \left( Q_i - Q_{i_0}\right) / \left( \rho c_p \right ) \] 
where $BET_i$ is the minimum moisture availability of a surface, $Q_i$ is the current water mixing ratio at the surface and $Q_{i_0}$ is that at a previous timestep.
Finally, ground flux contributions are calculated by solving the surface energy budget on each surface. The remainder, $G_i$ is the ground flux: 
\[ G_i = SW_i + LW_i - LH_i -H_i\]
where $H_i$ is the sensible flux, $SW_i$ is the net shortwave flux, $LW_i$ is the net longwave radiative flux, and $LH_i$ is the latent heating. 
Each model is the same for each land type, but the parameters $R, RW, W$ are assigned to each land type (Table~\ref{tab:urban_model}). 

\begin{table}
\begin{tabular}{lclll}
\toprule
Parameter & name & 1 & 2 & 3 \\
\midrule
Urban fraction &  \textsc{FRC\_URB} & 0.5 & 0.9 & 0.95 \\
Roof level &\textsc{ZR} &5.0&  7.5 & 10.0 \\
Roof height std. deviation & \textsc{SIGMA\_ZED} & 1.0 &  3.0 &  4.0\\
Roof width &  \textsc{ROOF\_WIDTH} & 8.3& 9.4 & 10.0 \\
Road width & \textsc{ROAD\_WIDTH} &  8.3&  9.4&  10.0 \\
\bottomrule
\end{tabular} 
\caption{WRF-Urban parameters which vary by urban land type 1-3, where 1 is low intensity urban land and 3 is high intensity land.}
\label{tab:urban_model}
\end{table}

We analyze the following variables from the model: 2-meter temperature (T2),
 land-use type (LU INDEX), vegetative fraction (VF),
 latent heat (LH), 
 sensible heat (HFX), ground flux (GRDFLX), 
incoming shortwave radiation (SWDNB),
reflected outgoing shortwave radiation (SWUPB), 
reflected incoming longwave radiation (LWDNB) and outgoing longwave radiation (LWUPB).  
%We compute the net energy flux $F$ as
%\begin{equation}
%%\label{eq:seb}
%F= r_{net} -LH - HFX - GRDFLX
%\end{equation}
%where the net radiation flux is the sum of the net radiative fluxes $$r_{net}=LWDNB - LWUPB+ SWNDB-SWUPB$$.

We compare the simulated 2-meter temperature against the temperature observation in the nearest grid cell, matched using the \texttt{ll{\_}to{\_}xy} function in the Python WRF library. 
From the 83 unique observation sites, we have 43 unique model grid cells. This means the remaining 40 sites are located in the same model grid cell as another site; on average, 4 sites are located in a single grid cell with multiple observation sites (a median of 3 sites, with a standard deviation of 2.3 sites per grid cell). The most number of sites from a single model grid cell is 9. In these cases, we compare both observations to the model grid cell, effectively oversampling those cells. 
We reproduce model results for all grid cells in the 40km$^2$ domain around Baltimore (not shown) and find that site selection within this domain does not affect results.% in the Supplemental Material. 
In all cases, we compare data taken or simulated between the periods of July 1, 2016 and August 30, 2016 and report our results in local time (GMT-4). 

\subsection{Evaluation Metrics}
\label{sec:methods_evaluation}
We evaluate the simulations using several statistical methods. 
First, the root-mean squared error $RMSE$:  
\begin{equation}%\label{eq:rmse}
RMSE = \sum_{i=1}^{N} \left(S_i - O_i\right)^2
\end{equation}
which compares observations $O_i$ to simulation data $S_i$.
%The $RMSE$ in this case has units of $^\circ$C$$. 
Next, we examine the Pearson correlation $r$ and its corresponding p-value $p$, indicating the probability that the given correlation is due to chance. The coefficient of determination, $r^2$. We also examine the Modified Index of Agreement or $MIOA$ (Willmott 1981), varying between \(0\) and \(1\), with \(1\) indicating a perfect match and zero indicating no match: 
\begin{equation}
MIOA = 1 - \frac{\sum_{i=1}^{N}(O_{i}-S_{i})^{j}}{\sum_{i=1}^{N}|(S_{i}-\bar{O})|+|(O_{i}-\bar{O})|^{j}}
%\label{eq:modified_index_of_agreement}
\end{equation}
where $j=1$. 
Then, we calculate the percent bias (PBIAS), where a value of zero is a perfect match; positive values indicate that the model overestimates the model while negative values indicate that the sensor is underestimates the reading.
\begin{equation}
PBIAS = \frac{\sum_{i=1}^{N}(O_{i}-S_{i})}{\sum_{i=1}^{N}O_{i}}*100
%\label{eq:pbias}
\end{equation}

%\suibsection{Statistics}
To separate temporal error from error induced by spatial variability, we calculate each of the above metrics for spatially average temperature and temporally averaged temperature. We analyze temporally averaged temperature and spatially averaged temperature taken from data at all hours, at 6am local time, and 4pm local time. 
% histograms metrics
To examine spatial variability, we compute histograms of spatially averaged temperature for each land type. Histograms are computed using fixed bin widths of $1^\circ$C, meaning that each data set is sorted into a different number of bins. We also report the sample mean $\mu$ and standard deviation $\sigma$ of spatially averaged temperature. 

To quantify spatial auto-correlation within the urban heat island, we calculate the semi-variance. The semi-variance provides a measure of spatial variance as a function of distance; it indicates an average difference between two data points $f(a), f(a+h)$ given their distance apart $h$: 
\begin{equation}
 s(h) = \frac{1}{2 N(h)} \sum _{N(h)} \left(f(a+h) -f(a)\right)^2
\label{eq:semivariance}
\end{equation}
This assumes that $h$ is a continuous variable and that it is possible to average over several points with distance $h$, which is impossible in practice. Thus, we calculate what is called the experimental semi-variance by making $h$ a discrete variable ranging from 1km to 25km with intervals of fixed-width distances (1 km for modeled data or .1 km for observed data). 

% UHI metrics
We also calculate the urban heat island intensity $\Delta T = T_u - T_r$, a temperature difference between a rural and urban area. We calculate this in two ways. First, we calculate $\Delta T$ using the average of sites or model grid points in rural forest land as our rural reference $T_r$. We then subtract this from each urban site's temperature to calculate $\Delta T$. 
Second, we calculate $\Delta T$ using a single rural reference site or model grid point. We then average over all sites within a land type to produce an estimate of $\Delta T$ for each land type. 

\subsection{Software}
Data was analyzed using Python version 2.7. We used the following libraries: numpy version 1.11.3, scipy version 0.19.0, pandas version 0.20.1, and WRF version 1.0.1. 
%(u'Pandas version: 0.20.1', 'Numpy version: 1.11.3', 'Scipy version : 0.19.0', 'xarray version: 0.9.5', 'wrf version: 1.0.1')
%The code is available at :  github.com/gottscott/xx

\section{Results}\label{sec:results} 
\subsection{Observations}
\label{sec:results_obs}
\subsubsection{Statistical distribution}

\begin{figure}
\centering
\includegraphics[width=\textwidth]{rnw_chapter4/figures/landcover_distribution.eps}
\caption{Distribution of 2-meter temperature for observations (left column) and model (right column) for all hours (top row), 6am (middle row), and 4pm (bottom row) by land type. }
\label{fig:hist}
\end{figure}

The average temperature for July through August 2016 
%ranged from 13.7-48.0$^\circ$C, with an average temperature of 
was 26.9$^\circ$C. 
The lowest temperature recorded was 13.7$^\circ$C and the highest temperature recorded was 48.0$^\circ $C; as this latter temperature was measured at 9am by a thermometer in full sun, a time when the iButton bias correction is low, we conclude that this was probably an erroneously high reading.
The coolest temperatures occured at 6am temperatures (on average, 23.6$^\circ$C) and the warmest temperatures occurred at 4pm (on average,  30$^\circ$C). 
Temperatures vary day to day and week to week: mean daily temperatures has a range of approximately $10^\circ$C for the study period, with average daily temperatures (spatially-averaged) varying between $21.1^\circ$C and $31.4^\circ$. The standard deviation of mean daily temperature is $2.2^\circ$C, showing that temperature varies by several degrees day to day. 
Temperature also varies across the city, with a range in time-averaged temperature across all sites of 5.5$^\circ$C and an overall standard deviation of 1.2$^\circ$C. Fig.~\ref{fig:hist}a shows the statistical distribution of time-averaged temperature for all hours at each site grouped by land type. 
Temperatures %are normally distributed for each land type, and 
for high intensity land use is hottest, with a mean temperature of 27.8$^\circ$C. Medium and low intensity land is the second hottest ($\mu = 27.15^\circ$C), and rural forests are coolest ($\mu = 24.44^\circ$C). While high intensity land is on average hotter than other less developed land types, there are many individual sites that are cooler than medium intensity land. Similarly, while medium intensity land is hotter than open space on average, there are many sites classified as open space which are hotter than medium intensity land, and while rural land is the coolest, there are open space sites which are just as cool. 

Variability in time-averaged temperature is small-the standard deviation of temperature is less than 1$^\circ$C for each land type, with open space being the most spatially variable ($\sigma = 0.89^\circ$ C). 
%We repeat this analysis for the statistical distribution of mean 6am temperatures in Fig.~\ref{fig:hist}b) and see that similarly, higher intensity land types are hotter than lower intensity land types, with high intensity land the hottest, and rural forest the least hot.
The same separation between land types happens for 6am and 4 pm temperatures (Fig.\ref{fig:hist}b).
 We next turn to 4pm temperatures in Fig.~\ref{fig:hist}b, and see again that higher intensity lands are hotter and that temperatures at 4pm are more variable than those at 6am, with medium and low intensity land types having the highest variability ($\sigma = 1.92$). 
Thus, temperatures vary by land type, with higher intensity land types  warmer than lower intensity land types, 
though there is significant overlap amongst the land types.
% Note: add urban forest or urban green category here? 

	\subsubsection{Spatial auto-correlation}
\begin{figure}
\centering
\includegraphics[width = .75\textwidth]{rnw_chapter4/figures/semivariogram.eps}
\caption{Semi-variograms showing distance versus the semi-variance (Eq.~\ref{eq:semivariance}) for observed temperature (top row) and modelled temperature (bottom row) at 6am (left column) and 4pm (right column). Dashed line indicates the sample variance. 
}\label{fig:semiv_obs}
\end{figure}
%Having seen how time-averaged temperature varies by land type, 
We next examine how different locations vary from one another by looking at the spatial auto-correlation, which measures the statistical similarity between two nearby points. Figure~\ref{fig:semiv_obs} shows a semi-variogram: the semi-variance (in units $^\circ$C$^2$) of temperature at 6am is plotted against distance. Here, we show data from all land types together. When we disagregate the data by land type, we find that results are inconclusive as the maximum distances are under 12km. 
For sites within one kilometer of one another, the semi-variance is less than $0.5^\circ$C$^2$, indicating that nearby sites are highly spatially auto-correlated and thus similar to one another. 
As distance between sites increases, the semi-variance increases linearly, indicating that farther away points are more and more dissimilar. While the semi-variance is noisier at longer distances, this is due to having fewer datapoints at these distances. 
The theoretical limit of the semi-variogram as distance $h$ increases is the sample variance, in this case $1.4^{\circ} \text{C} ^2$; the lack of data at longer distances could be why the semi-variogram exceeds the theoretical limit for some of the distances between 15-25 km. 
In Fig.~\ref{fig:semiv_obs}b, we plot distance versus semi-variance for 4pm temperature. We see that spatial autocorrelation is lower for all sites; for sites within one kilometer, the semi-variance is 1km, higher than that at 6am. The semi-variogram is noisier than that of 6am, with semi-variance increasing little with increasing distance until 15 kilometers, when it increases sharply. Even at 25 kilometers, semi-variance remains below the theoretical limit of the sample variance ($3.9^\circ \text{C}^2$).This indicates that spatial auto-correlation increases little with distance for distances under 15 km.
For both times, we see that sites closer together are more auto-correlated, that is, similar to closer sites, and farther sites are less auto-correlated and more dissimilar. This is more true for 6am temperature than for 4pm temperature. 

\subsubsection{Vegetation}

\begin{figure}
\centering
\includegraphics[width = \textwidth]{rnw_chapter4/figures/vegetation_fraction.eps}
\caption{Surface vegetation versus temperature for observations (left column) and model (right column) for all hours (top row), 6am (middle row), 4pm (top row): in a,c, and e, satellite NDVI versus mean observed temperature at each site, and b,d, and f, model vegetative fraction. The correlation $r$ and correlation p-value $p$ are labeled. Colors indicate each land type. N indicates the number of model grid points included in each land type.}
\label{fig:veg}
\end{figure}

We now turn to the cause of spatial variability in temperature. We have shown how temperature varies by land type; as one differing factor between land types is the presence of vegetation, we next quantify the relationship between temperature and green infrastructure in Baltimore. 
Because there is no common index for indicating the presence of green infrastructure directly, we first turn to satellite-measured NDVI, a proxy for photosynthetic activity that can indicate the presence of vegetation and is a continuous variable. 
In Fig.~\ref{fig:veg} we examine the relationship between NDVI and time-averaged temperature from data taken from all hours, from 6am data, and from 4pm data. We first see that the most developed land types have the lowest values of NDVI, and that the least developed types have higher values of NDVI, though there is a wide range of variability and overlap between the land types. For each time examined, there is a negative relationship between NDVI and temperature, indicating that vegetation (as observed from above by satellite) is related to cooler temperatures and  that variability in NDVI is an important factor in determining spatial variability of air temperature. We also see that the relationship is different at different hours in that the correlation between temperature and NDVI is strongest at 6am ($r = -0.68$, $p< 0.05$) and weakest at 4pm ($r= -0.21$, $p<0.05$). 

\begin{figure}
\centering
\includegraphics[width = \textwidth]{rnw_chapter4/figures/veg_elevation_corrected.eps}
\caption{Surface vegetation versus temperature for observations (left column) and model (right column) for all hours (top row), 6am (middle row), 4pm (top row): in a,c, and e, satellite NDVI versus mean observed temperature at each site, and b,d, and f, 0.01 times model vegetative fraction. The correlation $r$ and correlation p-value $p$ are labeled. Colors indicate each land type. N indicates the number of model grid points included in each land type.}
\label{fig:veg_elev_corrected}
\end{figure}

It is possible that elevation variability affects the observed  temperature-vegetation relationship. The most developed parts of Baltimore are located near the harbor at low elevation, while less developed sites with more vegetation are located at higher elevations. The elevation of our study sites ranges from 0 to 195 meters, with a median elevation of 35 meters and a mean elevation of 50 meters, indicating that the distribution is right-skewed and that most sites have relatively low elevations. Both morning and afternoon temperatures are negatively correlated with elevation: the correlation of 6am temperatures $T_{6am}$ with elevation is $-0.67$, and the correlation of 4pm temperatures $T_{4pm}$ with elevation is $-0.45$. To separate the influences of these factors on temperature, we perform a multiple linear regression. The results show that 
\[ T_{6am} = -0.015\cdot \text{elevation} -2.91 \cdot \text{NDVI} + 24.68\] 
\[ T_{4pm} = -0.017\cdot \text{elevation} +0.195 \cdot \text{NDVI} + 32.4\]
where elevation is in meters and NDVI is unit-less. 
This regression result indicates that 6am temperature decreases by $0.015^\circ$C for every meter increase in elevation ($0.017^\circ$C/m for 4pm temperature). Similarly, temperature decreases by $0.29^\circ$C for every increase of $0.1$ in NDVI. 
These coefficients, when multiplied by the relevant quantity (NDVI or elevation), are similar in magnitude. 
This suggests that at high elevations, the presence of low temperatures at the highest elevation sites is explained primarily by elevation.
 For example, at the highest elevation site, 195 meters, NDVI is $0.5$, meaning that cooling due to elevation is $2.9^\circ$C while cooling from NDVI is estimated at $1.5^\circ$C. Thus, at this site, elevation causes more cooling than does vegetation. 
 Similarly, at low elevations the reverse is true: at the lowest elevation site, located at sea level, NDVI is $0.3$, meaning that elevation cooling is null whereas the NDVI cooling is  $0.9^\circ$C. 
To systematically compare these effects, 
we compute elevation-corrected temperatures, $T_{6am} + 0.015\cdot \text{elevation}$ and $T_{4pm} + 0.017\cdot \text{elevation}$, and compare these to NDVI in Fig.~\ref{fig:veg_elev_corrected}. We see that while $T_{6am}$ is negatively correlated with NDVI ($r=-0.61$, $p<0.05$), the correlation between NDVI and $T_{4pm}$ is no longer significant ($r=-0.04$, $p=0.7$). 
Thus, we conclude that higher levels of vegetation are linked with lower 6am temperatures in Baltimore, but not for 4pm temperatures. This suggests that satellite vegetation is linked with cooling of $2.9 - 0.015\cdot \text{elevation} ^\circ$C/$\text{NDVI}$ at 6am. As NDVI has a maximum of $0.6$ in our study, this means that the maximum theoretical cooling at 6am from vegetation is $1.74^\circ$C. 

\begin{figure}
\centering
\includegraphics[width=\textwidth]{rnw_chapter4/figures/diurnal_sub1km.eps}
\caption{Diurnal variability for sub-kilometer scale vegetation types: (a) street trees in sites dominated by impervious landcover, (b) urban parks with trees, (c) open fields in urban parks, and (c) urban forests. }
\label{fig:diurnal_urbanforests_etc}
\end{figure}

Our land use categorizes sites based on satellite data at a scale of 1km.  Sub-kilometer spatial variability is exists as green infrastructure in the form of the presence or absence of local vegetation, including street trees, small parks, and low vegetation (grass, bushes, or other low plants). As large-scale vegetation affects temperature, it is plausible that smaller-scale vegetation does as well. 
Thus, we now turn to local site vegetation whose variability is too small to be captured by satellite sensors. 
In Fig.~\ref{fig:diurnal_urbanforests_etc}, we look at diurnal temperature variability for local site vegetation characteristics. To separate large scale and local scale variability, we separate by land types. 
At sites with impervious landcover and street trees (Fig.~\ref{fig:diurnal_urbanforests_etc}a), %daytime high temperatures in high intensity sites are lower than those shown in Fig.~\ref{fig:diurnal}a, with similar low temperatures. 
daytime high temperatures in high intensity sites are lower than the daytime high temperature averaged for all high intensity sites (compare Fig.~\ref{fig:diurnal_urbanforests_etc}a to Fig.~\ref{fig:hist}a).
There are also differences in the response to street trees between land types: 
medium and low intensity land is similar to high intensity land, with open space a few degrees lower. 
Next, we examine sites with grass and trees (Fig.~\ref{fig:diurnal_urbanforests_etc}b), which include urban parks as well as greened developed spaces. These are a little cooler during the day and no different at night than high intensity sites with street trees, which shows that adding grass makes little difference for high intensity sites. 
This may be because grass in high intensity land exists as a small patch, because of less available area, whereas
whereas grass in an open space was often in a larger field. 
This is in contrast with open spaces, which are cooler at night for sites with grass and trees than with street trees alone, particularly at night. Next, we examine sites with grass and no trees located in parks; these include baseball and football fields as well as a golf course driving range (Fig.~\ref{fig:diurnal_urbanforests_etc}c). 
This type of green infrastructure shows significant differences for all land types: daytime temperatures are the highest of all green infrastructure types examined, while nighttime lows are 1.2-1.3$^\circ$C degrees cooler than for other land modifications. 
Finally, we examine sites within the canopy of urban forests (Fig.~\ref{fig:diurnal_urbanforests_etc}d). 
We find that these are coolest during the day for all land types, though at night show little difference from sites with less vegetation (namely, sites with street trees or in parks with trees) and are in fact warmer than open fields in parks. 
We note that sensors had more sun exposure during the daytime in open sites as compared with urban forests, possibly explaining results. As more sun exposure results in more net radiation and thus warmer temperatures, it is difficult to entirely disentangle this phenomena. 
Thus, in addition to land use, vegetation variability at sub-kilometer variability affects temperature, and the cooling intensity and timing linked with green infrastructure differs by green infrastructure type. 

\begin{figure}
\centering
\includegraphics[width=\textwidth]{rnw_chapter4/figures/distance_to_park.eps}
\caption{Distance to park versus time-averaged (a) $T_{6am}$ and (b) $T_{4pm}$. Each dot represents one site.}
\label{fig:park_distance}
\end{figure}

In Fig.~\ref{fig:semiv_obs}, we saw that urban temperature is spatially auto-correlated at both night and day, and that vegetation is linked to cooler temperatures at several scales. One question relevant to decision makers is how far the cooling benefit of parks and trees persists. That is, if two sites are close by, and one site is a site with green infrastructure causing a lower temperature, how far does that cooling benefit extend? 
We address this question in Fig.~\ref{fig:park_distance}, which shows the distance from the nearest park to each site versus its mean temperature for $T_{6am}$ and $T_{4pm}$ for sites within 500 meters of a park. We see that while $T_{6am}$ is positively correlated with park distance ($r=0.52$, $p< 0.05$), $T_{4pm}$ shows no significant relationship ($p>0.05$).  We quantify the cooling by distance using a linear regression and find that cooling benefits decay outside of parks by 0.004$^\circ$C/meter. 
Correcting for elevation does not significantly change this finding; no significant relationship is found between elevation corrected $T_{4pm}$ and park distance, and elevation corrected $T_{6am}$ increases outside of parks by 0.004$^\circ$C/meter.  
This suggests that parks can offer nighttime cooling benefits to surrounding neighborhoods, though this cooling benefit is overall small--for example, living within 100m of a park is approximately $0.5^\circ$C cooler than living farther away. Any daytime cooling benefits are unclear beyond park boundaries. 

\subsubsection{Urban heat island intensity}
%How robust is this result? The very next figure seems to show that it's extremely sensitive to our selection of the rural sites--if we'd had one more site in a field or open lawn it might have flipped. You've also shown that its sensitive to urban site distribution, but at least for those you have a large number that you can average across. 
%
%I'm not arguing that we have to throw up our hands, but if, for example, there is a systematic difference between forest and field as the rural comparison then why not lead with that fact rather than opening with a figure that might have gone the other way with one more or one fewer site of either type?
%
%And while I'm on the subject:
%
%1. It feels like there's something potentially interesting in the fact that the sites resembling WMO standards--i.e., those that are generally open, grass-covered environments--show higher nighttime/morning UHI, while those under forest cover show close to the opposite. Perhaps something to comment on w.r.t. our understanding of UHI phenomenology?
%
%2. To what extent do you think these results are driven by shading?

We previously saw in Fig.~\ref{fig:hist} how high intensity land is the hottest land type at both 6am and 4pm.  In Fig.~\ref{fig:diurnal}a, showing hourly temperature averaged across sites in each different land type, we see that this is true for all hours. That is, high intensity land is the warmest land type at every hour, showing that the urban microclimate varies significantly within the city. 
Medium and low intensity land types are next hot, remaining within 1$^\circ$C of high intensity land types throughout the day. Open space is next coolest, with a minimum temperature of $22.5^\circ$C and a maximum temperature of $28.9^\circ$C. 
Though open spaces are cooler than high intensity land, these sites have a diurnal range of $6.4^\circ$C, similar to that of high intensity land types ($6.9^\circ$C), indicating its ability to both warm up during the day and cool off at night. 
The coolest sites throughout the day are located in rural forests. 
%Urban forests retain heat at night compared to rural forests, remaining hotter than medium and low intensity land types at 6am, but are much cooler during the daytime, staying cooler at 4pm than all developed land types.
The temperature differences between sites give rise to a temperature difference, $\Delta T$, between urban, developed sites and the rural sites. We first compute the difference between the mean of each land type and the mean of the rural sites. 
In all land types, this temperature difference is largest in the early morning hours, with a maximum at 9am, and diminishes throughout the day, attaining a minimum at 8pm. The average $\Delta T$ ranges from $1.8^\circ$C to $3.4^\circ$C
and is highest for high intensity land throughout the day. Open space and rural forests have the lowest temperature difference with respect to rural observations. Open space $\Delta T$ remains fairly constant throughout the day, with a slight increase at 6am and subsequent decrease at 9am, indicating an offset in diurnal temperature cycle timing compared to rural forests. It is possible this is attributable to faster heating in open areas due to more open geometry and less shading. 

\begin{figure}[h]
\centering
\includegraphics[width=\textwidth]{rnw_chapter4/figures/diurnal.eps}
\caption{Average temperature (top row) and temperature difference (bottom row) by hour for observations (left column) and model (right column) for each landcover type. N indicates the number of model grid points included in each land type. }
\label{fig:diurnal}
\end{figure}

\begin{figure}[h]
\centering
\includegraphics[width=\textwidth]{rnw_chapter4/figures/DT_sensitivity.eps}
\caption{Top row: average temperature difference between the average of sites in each urban land type and various rural sites for observations (top row): a) BWI airport , b) site X, c) site Y  d) site Z. Bottom row: same as above but for model. N indicates the number of model grid points included in each land type.
}
\label{fig:dt_sensitivity}
\end{figure}

Above we have shown how variability within a city affects $\Delta T$. But rural areas are not monolithic either, so we test the sensitivity of $\Delta T$ to rural station selection  (Fig.~\ref{fig:dt_sensitivity}) by varying rural sites.
%The temperature differences between sites give rise to a temperature difference, $\Delta T$, between urban, developed sites and the rural sites (Fig.~\ref{fig:dt_sensitivity}).
 We first compute $\Delta T$ with  respect to a nearby ASOS station used in a previous Baltimore study \citep{li2013synergistic}. We see in Fig.~\ref{fig:dt_sensitivity}a that
%We find significant differences in both the timing and intensity of the diurnal cycle of $\Delta T$: 
in all land types, the temperature difference is largest at night and through the early morning hours.
$\Delta T$ diminishes shortly after 6am, with differences at noon being small or negative: while the average 6am $\Delta T$ value  is $1.4^\circ $C, noontime urban-rural differences diminish to 0.0-0.2$^\circ$C in high and medium and low intensity land types, and in open spaces, noontime $\Delta T$ becomes negative.
Temperature differences with respect to rural forests are negative throughout the whole day, though exhibit a similar pattern to other land types in that $\Delta T$ decreases during the day and increases at night and in the early morning. The diurnal range of $\Delta T$ averaged across all land types is $3^\circ$C. 

%\begin{table}
%\begin{tabular}{lrrrllll}
%\toprule
%Site &   Elevation &      NDVI &         VF &  Sun or shade & Attachment & Landcover &  Notes \\
%\midrule
%68  &  195.133606 &  0.527851 &  81.528183 &      shade &       tree &          grass &                               Darryn's house \\
%75  &  176.847046 &  0.457631 &  74.932465 &      shade &       tree &           dirt &                                Ben's house \\
%103 &   86.993736 &  0.524227 &  66.301079 &    partial &       tree &          grass &          Meadowood park behind rain garden \\
%130 &   84.259598 &  0.450265 &  79.935478 &    partial &       tree &          grass &  Cherry tree in middle of grounds entrance \\
%\bottomrule
%\end{tabular}

\begin{table}
\begin{tabular}{lrrrlll}
\toprule
Site &   Elevation &      NDVI &         VF &  Sun or shade & Attachment & Groundcover \\
\midrule
68  &  195.1 &  0.53&  81.5 &      shade &       tree &          grass  \\
75  &  176.8 &  0.46 &  74.9 &      shade &       tree &           soil or leaf litter\\
103 &   87.0 &  0.52 &  66.3 &    partial &       tree &          grass\\
130 &   84.2 &  0.45 &  79.9 &    partial &       tree &          grass \\
\bottomrule
\end{tabular}
\caption{Rural site characteristics.}
\label{tab:rural_ibuttons}
\end{table}

We next examine $\Delta T$ calculated using site 68 (Fig.~\ref{fig:dt_sensitivity}b), the northernmost and highest elevation site in our study. This site is located in a forested residential yard abutting a forest and has an NDVI of $.5$ (Tab.~\ref{tab:rural_ibuttons}), a high value as the maximum NDVI in this study is $0.6$. At this site, $\Delta T$ is greater for higher intensity land types. Minimum $\Delta T$ occurs at 4am, increases throughout the day, and is largest in the early evening, though average $\Delta T$ varies by only by $1.3^\circ$C throughout the day.
Next, we examine site 75, the second-highest site located under tree canopy cover in a residential neighorhood; NDVI at this site is 0.45. 
$\Delta T$ at this site is smallest in the morning and largest in the evening. The diurnal range is $2^\circ$C and $\Delta T$ is positive for all land types except rural forests. 
Site 103 (Fig.~\ref{fig:dt_sensitivity}d) is located in an open field, outside of tree canopy cover in a suburban park.  
This site has the largest diurnal range of $\Delta T$ for all rural sites examined ($3.9^\circ$C); the diurnal range is highest at night and lowest during the day and evening, with maximum and minimum occuring at 6am and 7pm.
While the cycle is similar to that of the BWI site, the daytime $\Delta T$ is significantly higher, and the minimum $\Delta T$ occurs much later. 
Finally, at site 130  (Fig.~\ref{fig:dt_sensitivity}e), located on the edge of a forest and within 300 meters of a lake, $\Delta T$ varies little throughout the day, changing by only $1.2^\circ$C. Maximum $\Delta T$ occurs at 8am and minimum $\Delta T$ occurs at 4pm. 

At all of these rural sites, there is evidence of a pronounced UHI at some point during the day or night, though when this occurs varies.
 For three out of five sites used as a rural reference site, $\Delta T$ is highest at night and the early morning, with the highest $\Delta T$ occurring between 6am and
9am.  
Accordingly, daytime $\Delta T$ is lower or negative during the day at these sites. 
At the other two sites, the opposite is true- $\Delta T$ is lower in the morning and higher during the day and at night. 
The sites which have higher daytime $\Delta T$ (68, 75) are located under thicker canopy cover. The other sites (BWI, 103, 130) are located in open environments. These sites are also cooler at night. 
Thus, our result suggests that in addition to urban land type, the rural microclimate significantly impacts the timing and magnitude of $ \Delta T$. We conclude that for forested rural sites, $\Delta T$ is highest during the afternoon and evening, while for open rural sites, $\Delta T$ is highest at night. 

%Rural areas are not monolithic, so we test the sensitivity of $\Delta T$ to the rural reference by examining a nearby ASOS station that has been used in previous Baltimore studies \citep{li2013synergistic} (Fig.~\ref{fig:diurnal}). We find significant differences in both the timing and intensity of the diurnal cycle of $\Delta T$: the largest values of $\Delta T$ occur at 6am and the smallest values occur at noon. This is true for all examined land types. In high and medium intensity land types, urban-rural differences diminish to 0.0-0.2$^\circ$C at noon, and in open spaces, noontime $\Delta T$ becomes negative. 
%Rural forests are always cooler than the station, meaning that temperature differences with the forested sites are largest during the day and most similar during the nighttime hours. 
%The BWI station is classified as open, developed space according to the land type classification and is located at the airport; the comparison with rural sites suggests that it may not be a true rural site in the physiological sense. 
%This demonstrates that key parameters of the urban heat island are sensitive enough to rural site selection to alter key parameters such as the timing and magnitude. 
%
% consider possibly moving this to a separate figure/paragraph? 

%possible figure about distance to park? 
\subsection{Observations discussion}
We have examined Baltimore's urban heat island using a dense network of thermometers and seen that air temperature's spatial variability is around $1^\circ$C. Significant differences in temperature are attributed to land use: we find high intensity land to be the hottest and rural forest land the coolest. That is more developed land is hotter in Baltimore. We also find that temperature exhibits spatial auto-correlation that increases linearly with distance, and doesn't appear to reach a maximum within the distances examined in our observation network ($<25$ km). 
 Our result suggests that rather than air temperatures being entirely random, temperature in one location affects nearby locations, particularly for 6am temperatures. Furthermore, statistical models that rely on the semi-variogram to model urban temperature (e.g., Krieging) often fit semi-variograms to a few statistical models (circular, exponential, etcetera) that assume that the semi-variance reaches its sill at the sample variance (\textit{eg}, \cite{hardin2017urban}). 
 Our results suggest that such models may make incorrect assumptions about spatial auto-correlation within the urban heat island and so may be better at capturing larger scale gradients than temperature variability within the urban core.
 
 % Can you site examples of this, and state why it matters? Also, as we've discussed, isn't this very sensitive to your sampling density and extent outside of the city?
  
%As statistical models like Kreiging model spatial auto-correlation by fitting observed semi-variograms to statistical models of semi-variance that attain a maximum value or sill within the domain of interest, this has the added implication that such techniques may fail as models of urban temperature. 

Our result associates lower temperatures with the presence of vegetation but indicates that the strength of the relationship depends on the hour of day. Cooler temperatures are linked with satellite-derived vegetation in the morning, but not in the afternoon.  We also find that sub-kilometer scale variability in vegetation or green infrastructure also affects temperature: the presence of local trees, grass, or other vegetation is linked with cooler temperatures. Again, the sign and strength of temperature's relationship with green infrastructure depends on the time of day examined. 
% explain discrepency here

We find that urban forests provide the most daytime cooling, and that open areas filled with grass or other low vegetation provide the most nighttime cooling. 
In our study, street trees are the least effective at cooling:  sites with street trees do have slightly lower 4pm temperatures than areas with no green infrastructure, but this difference is small ($30.8^\circ$C versus $31.2^\circ$C). 
We note that most street trees in the areas examined are recently planted (within the last decade), meaning that cooling benefits may yet be realized. 
Additionally, nighttime but not daytime temperature decreases with distance from parks, suggesting that it is possible for green infrastructure benefits to extend beyond their immediate footprint. 

Our results raise the question of whether daytime cooling from greening is possible. While we see this relationship for sub-kilometer scale vegetation, we do not see it in satellite-observed NDVI. 
As discussed in \cite{scott2017temperature}, NDVI represents green intensity seen from above rather than ground-level vegetation. 
The discrepancy with smaller-scale vegetation could be interpreted as a function of scale: for example, the daytime cooling benefit of green infrastructure could simply be highly localized. 
Vegetation affects temperature in a number of ways, from increased evapo-transpiration, increased shading, to albedo modifications, so it is possible that the NDVI-temperature relationship may reflect different characteristics than as with local green infrastructure. 
These results suggest that greening strategies that enhance parks, trees or vegetation, enacted both on scales of 1km or smaller scales (<1km) have the potential to cool the city, but the intensity of the cooling and whether the city experiences nighttime or daytime cooling depends on the greening strategy. Our results identify urban forests as the green infrastructure the most effective option for daytime cooling and parks with open fields as the most effective for nighttime cooling. 
While a full investigation of this phenomena is beyond the scope of this study, our results suggest that the forests offer transpiration and shading, leading to daytime cooling, while the open areas allow for nighttime radiative cooling, greater energy loss, and thus greater cooling. 

We observe that the intensity of urban-rural temperature differences varies significantly within the urban heat island; we attribute this variation to land use intensity and find that more developed land types have higher values of $\Delta T$.  
Additionally, the timing and intensity of these differences is sensitive to the rural site used as the rural comparison.
Urban-rural differences are highest in the night and morning and lowest in the evening or daytime when rural, open sites are considered, but when rural, forested sites are examined, these differences are highest during the day. 
Thus, the timing, intensity, and even sign of $\Delta T$ is very sensitive to both urban and rural station. 
%Compared with our rural measurements,  $\Delta T$ remains high during the day and reaches a minimum at 8pm and a maximum at 9am. This is in contrast with using the BWI station as a rural reference (as done in \citep{li2013synergistic}), which shows  a negligible daytime UHI at noon and a maximum $\Delta T$ of several degrees at 6am. 
It has been previously reported that urban microclimate has a significant effect on the intensity of $\Delta T$ \citep{lcz};  our result shows that the definition of a rural site or sites can be just as critical when characterizing details of the UHI.
This has ramifications for previous studies: in \cite{scott2018reduced}, urban-rural temperature were shown to decrease during a range of warmer conditions, including in Baltimore. 
%Authors performed sensitivity analysis by varying rural stations, though the analysis was restricted to daily minimum and daily maximum temperatures at rural stations far from urban centers.
%While our study does not examine this question, our results do show that the timing and intensity of $\Delta T$ depends heavily on rural station selection, especially for rural sites within 25 km of the urban core. 
 Because our results show that it is possible for hourly temperature measurements to change  
$\Delta T$, the conclusions in \cite{scott2018reduced} may not hold for all of the sites examined in this study, though more analysis needs to be done to understand the role of rural station selection as our rural stations are not the same.

Our study confirms that a pronounced air temperature urban heat island exists throughout the city of Baltimore, and the magnitude of the UHI is higher in more developed areas.
That higher intensity land exhibits higher values of $\Delta T$ confirms that land use can drive temperatures within the urban heat island, as previously found in \cite{Huang20111753,Boug10}. 
 Studies examining the spatial extent of the UHI have found significant variability in LST \citep{Huang20111753}, but lower temperature variability in minimum daily air temperatures. 
Our results confirm the findings in \cite{scott2017intraurban} and demonstrate that there is lower variability in air temperature than in LST throughout the day. 

\subsection{Model}
\label{sec:results_model}
%We have quantified how temperature varies throughout Baltimore's urban heat island and identified a number of UHI characteristics of interest to scientists, planners, and policymakers.
%% A logical question is how might researchers in other cities replicate this process in order to develop an understanding of their urban heat island, given that most cities lack access to a network of temperature sensors. 
%Numerical models can aid UHI researchers in developing a process-based understanding of the urban environment, answer scientific and policy questions, and finally, diagnose possible solutions. This makes such models a powerful tool, and leads us to examine how well one such model, WRF, captures the important aspects of Baltimore's urban heat island identified in the previous section. 
%In all cases, we compare 2-meter temperature observations with those calculated by the model. 

\subsubsection{Evaluation Statistics}

\begin{table}
\centering
%Time error: 
\begin{tabular}{lrrrrr}
\toprule
{} &  all data &     6am &   16pm &   min &  max \\
\midrule
rmse        &      4.60 &  4.15 & 6.04 &  3.94 & 2.87 \\
correlation &      0.89 &  0.84 & 0.76 &  0.82 & 0.90 \\
p-value     &      0.00 &  0.00 & 0.00 &  0.00 & 0.00 \\
pbias       &      0.78 & -6.38 & 2.55 & -5.76 & 3.31 \\
mioa        &      0.93 &  0.83 & 0.85 &  0.82 & 0.92 \\
r\_squared   &      0.80 &  0.71 & 0.58 &  0.67 & 0.81 \\
\bottomrule
\end{tabular}
\caption{Model error due to temporal variability: evaluation of model against observations for time-mean temperatures for root mean squared error $RMSE$, correlation  $r$, correlation p-value $p$, percent bias $PBIAS$, mean index of agreement $MIOA$, and coefficient of determination $r^2$ for 6am, 4pm, $T_{min}$ and $T_{max}$.  }
\label{tab:time_error}
\end{table}

\begin{table}
\centering
%Space error: 
\begin{tabular}{lrrrrr}
\toprule
{} &  all data &     6am &   16pm &   min &  max \\
\midrule
rmse        &      0.93 &  3.32 & 3.70 &  2.74 & 6.09 \\
correlation &      0.64 &  0.52 & 0.46 &  0.54 & 0.40 \\
p-value     &      0.00 &  0.00 & 0.00 &  0.00 & 0.00 \\
pbias       &      0.78 & -6.38 & 2.55 & -5.76 & 3.31 \\
mioa        &      0.69 &  0.55 & 0.45 &  0.57 & 0.42 \\
r\_squared   &      0.41 &  0.27 & 0.21 &  0.29 & 0.16 \\
\bottomrule
\end{tabular}


\caption{Model error due to spatial variability: evaluation of model against observations for temperatures averaged over space at each hour for root mean squared error $RMSE$, correlation  $r$, correlation p-value $p$, percent bias $PBIAS$, mean index of agreement $MIOA$, and coefficient of determination $r^2$ for 6am, 4pm, $T_{min}$ and $T_{max}$.  }
\label{tab:space_error}
\end{table}

We evaluate the WRF numerical simulations with respect to 2-meter atmospheric temperature. In order to assess how well the model reproduces the day-to-day temporal variability, we first compare spatially-averaged temperature observations to spatially-averaged modeled temperature. 
Table~\ref{tab:time_error} reports these results for several statistics for data from all hours, 6am, 4pm, $T_{min}$, and $T_{max}$.  
These statistics suggest that the temporal error in our simulations, that is, error caused by temporal variability in synoptic weather patterns, is somewhat high---for data from all hours, the RMSE (4.6) is slightly outside of the upper range of $4^\circ$C reported by the literature \citep{kim2013evaluation}, though this range includes means of daily data in addition to hourly data.
% add these values here. 
The modeled data is, however, highly correlated with observations ($r=0.89$, $p<0.05$) and has a low percent bias ($pbias = 0.78$). The mean index of agreement is close to one ($mioa=0.93$), indicating good agreement between hourly model and observational data.
Compared with all hours, the RMSE for 6am data is lower ($rmse= 4.15$) than that for all hours, suggesting that simulations at 6am better match observations. 
Other statistics indicate more variance in the 6am model performance: the correlation is lower ($r=0.84$), the index of agreement is lower ($mioa=0.83$), and the percent bias has a large absolute value ($pbias=-6.38$), indicating that the model underestimates 6am temperature with respect to observations. 
%
At 4pm, error statistics show worse model agreement is worse than at 6am,  %The error statistics for 4pm data show worse model agreement for than that for hourly data and 6am data,
and 4pm data has the highest mean error out of any category we examine ($rmse=6.04$). The correlation is similarly low ($r=0.76$), though both the percent bias and index of agreement show less variability than similar statistics for 6am data ($pbias=2.55$, $mioa=0.85$). This indicates that while overall statistics show that our simulation is in line with reported error, statistics at specific hours show worse agreement. Furthermore, the model both underestimates temperature in the morning and overestimates temperature in the afternoon. %Bias is particularly strong during the daytime hours. 

It may be possible that observations and the model have slightly altered diurnal cycles due to irregular canyon geometry in the observations. To allow for the possibility of this in our model evaluation, we also examine error statistics for minimum and maximum daily temperatures $T_{min}$ and $T_{max}$. The RMSE for $T_{min}$ is $3.94$, lower than of that for 6am, and the percent bias has a lower absolute value ($pbias=-5.67$), indicating better agreement and less variability in agreement when the diurnal cycles are shifted. However, the correlation is slightly lower ($r=0.82$) and the mean index of agreement is very similar ($mioa=.82$), suggesting that this improvement is minimal. The mean error for $T_{max}$ is the lowest mean error reported ($rmse=2.87$); similarly, the correlation is the highest reported ($r=0.9$). The percent bias is higher than that for 4pm data ($pbias=3.31$), and the index of agreement is similar to that for 4pm data ($mioa=0.93$).
Thus, while maximum and minimum statistics show some indication of a different diurnal cycles between observations and modeled data, they generally agree with hourly error statistics in sign and magnitude and so we conclude that a shifted diurnal cycle does not account for most of the error.

Next, we 
%compare time-averaged observations to time-averaged model data in the nearest gridpoint to 
assess how spatial variability in the model may contribute to error (Table~\ref{tab:space_error}). Overall spatial error is less than a degree ($rmse=0.93$) and percent bias is low ( $pbias=0.78$), though so is the correlation and corresponding coefficient of determination ($r=0.64$, $r^2=0.41$), indicating scatter relative to variability. 
%non-linearity in the model-observation relationship. 
The mean index of agreement is lower than that for temporal error, however, indicating worse agreement. 
As with the temporal error, the spatial error is higher for data at selected hours of the day than for all hours. For 6am data, $rmse=3.32$ and $pbias=-6.38$, indicating the model underestimates temperature by several degrees. 
%While this is an improvement over the 6am temporal error, $r=0.5$ (and correspondingly, $r^2 = 0.25$), indicating significant variability over space. 
Error statistics for 4pm data show poorer model agreement than for 6am data, with a higher RMSE ($rmse = 3.7$) and lower correlation ($r=0.36$) and corresponding coefficient of determination ($r^2 = 0.21$). The percent bias has a smaller absolute value, indicating that the model overestimates 4pm temperature ($pbias=2.55$). 
These statistics suggest that the model captures the internal spatial variability of the urban heat island for all hours, but struggles to capture these dynamics for hourly data. 
We compute error statistics by matching observation sites to the closest grid points rather than interpolating may introduce additional error, which may mean that these error statistics may underestimate model performance.

We also address the possibility that an altered diurnal cycle affects spatial error by calculating error statistics for $T_{min}$ and $T_{max}$. We see that error statistics for $T_{min}$ show slightly better agreement than for 6am data: the RMSE is slightly lower ($rmse = 2.74$), the correlation is slightly higher ($r=0.54$), the percent bias has a lower absolute magnitude ($pbias=-5.76$), and the index of agreement is higher ($mioa=0.57$). By contrast, error statistics for $T_{max}$ show poorer agreement than for 4pm data: 
the RMSE and bias are higher ($rmse=6.09$, $pbias=3.31$) and the correlation, index of agreement and coefficient of determination are lower ($r=0.4$, $mioa=0.42$, $r^2=0.16$). 
This indicates that while morning processes may occur at the incorrect time, the model incorrectly captures the timing of $T_{max}$ and overestimates its magnitude. However, this error is similar to what is reported in the literature. 

 %That error 
%%The spatial error is better than the temporal error, which is slightly higher than reported errors in the literature, 
%%generally show better model-observation agreement than do the error statistics for temporal error 
%suggesting that the model struggles to capture the observed time-varying temperature patterns, it performs better at reproducing the internal spatial structure of the urban heat island.
%%%note: contextualize these values. rmse = accuracy, pbias= spread/variance, mioa = ???
%%%
% Ben: 
%Is this synoptic? It's possible that the model error is due to missing synoptic scale events, but that would seem unlikely given the size of the WRF domains. Seems like it would be simpler just to say "time-varying."
%
%But that said, what does this sentence really tell us? Are the spatial and temporal metrics really suited to compare to each other? It seems to me that they're fundamentally different problems and that they have different populations for calculating statistics,  so just because one has a higher r or lower RMSE than the other doesn't mean that the model is doing "well" on one and "poorly" on another.

\subsubsection{Statistical distribution}
Using summary statistics is one way to evaluate model performance. %Having seen that these statistics suggest performance in line with that reported by the literature, 
We now turn to a detailed evaluation according to the important characteristics of Baltimore's urban heat island. 
%Figure~\ref{fig:hist}b shows histograms of spatially averaged temperatures at each site grouped by land types.
% note: calcluate this 
First, we examine the statistical distribution of time-averaged temperatures at each site for data averaged over all hours by land type (Fig.~\ref{fig:hist}b). 
As in observations, higher intensity land types are hotter than lower intensity land types, with high intensity land hottest, and rural forests coolest. High intensity land is significantly warmer than other land types and unlike observations, its distribution does not overlap with other land types, though less developed land types do have overlapping distributions. 
The distributions are less variable than observations, with standard deviations for all land types of $0.2^\circ$C or less. This suggests that the model underestimates the spatial variability present in observations, though as the model draws many key parameters from land-cover based lookup tables rather than fully distributed maps as well as averages over each numerical grid cell, this is not unexpected. 

Distributions of mean 6am modeled temperatures (Fig.~\ref{fig:hist}) resemble those at all hours in that the distribution of high intensity land does not overlap with other land types. Additionally, more developed land types are hotter than less developed land types. The notable exception to this is rural forests, which on average are warmer than open space as well as medium and low intensity land types. Variability is highest for rural forests ($\sigma = 0.4$), and for all land types, modeled variability in mean temperature is lower than of that for observations. Distributions of mean 4pm temperatures (Fig.~\ref{fig:hist}) for high intensity land remain warmest, and show more overlap among land types. These distributions are right-skewed, and in all cases, warmer than observations. Similarly, variability is much lower in the model than in observations. As with overall statistics of spatial variability, this may not be surprising as observations come from a variety of micro-climates whereas models inherently average sub-gridscale variability. Thus, these comparison of the distributions of observed and modeled temperature show that there are significant differences between models and observations. Modeled temperature is colder in mornings, and warmer in the evenings than observations, while exhibiting little of the variability within or between land types. 

\subsubsection{Spatial auto-correlation}
%We have shown how modeled temperature has low spatial variability (<$1^\circ$C) compared with observations. 
We next examine spatial auto-correlation in the model and how it changes as a function of distance. Fig.~\ref{fig:semiv_obs}c shows semi-variograms (semi-variance versus distance) for data at 6am. The semi-variogram has a nugget of approximately $0.5^\circ$C, indicating that points within one kilometer or less of each other vary by a little under 1$^\circ$C. Semi-variance increases with distance at a linear rate, and it is unclear if the semi-variogram reaches a sill or not. This indicates that temperature within the modelled urban heat island varies by $1^\circ$C or less. This is similar to the observed semi-variogram in Fig.~\ref{fig:semiv_obs}a, where semi-variorance appears to increase with distance beyond 20km. The semi-variogram for 4pm modeled temperatures (Fig.~\ref{fig:semiv_obs}d) shows that semi-variance remains constant or slightly increases with distance, indicating that modelled point similarity increases constantly with distance. The nugget is $1^\circ$C, indicating that there is sub-gridscale variability causes neighboring gridpoints to vary by about $1^\circ$C. 
% comparison with observation 
This shows that modeled semi-variograms have similar shapes to the observed semi-variograms, indicating that by this metric, the model is accurately capturing spatial auto-correlation in urban temperature. %Thus, we conclude that errors in spatial variability 

\subsubsection{Vegetation}
Observed temperature shows a negative correlation with satellite-measured vegetation NDVI. In the model, vegetation is represented by a parameter called vegetative fraction (VF) which varies between zero and one and comes from a lookup table.% [insert details from Ben/Hamada here]. 
While NDVI is not explicitly a fraction of vegetation in that it measures photosynthetic activity rather than extent, NDVI is comparable to VF in the qualitative sense that they both express a greenness fraction. Observed NDVI ranges from In the model, $VF$ ranges from 60\% to 90\% or is set to zero. While each land type has a variety of vegetative fractions, the model is not expressing the range of variability seen in observations in low-vegetation environments. This is particularly true for higher intensity land types: for example, high intensity land either has VF of either 0 or 0.6. As a result, the correlation between NDVI and VF is poor: $r=0.25$. 
We compare modeled temperature to vegetation fraction $VF$ in Fig.~\ref{fig:veg}b for temperature at all hours and see that
the model captures the observed negative relationship between temperature and vegetation: $r= -0.49$, similar to observations. Next, we examine the hourly data and see that 6am temperature is slightly less correlated with vegetation than observations ($r= -0.59$, Fig.~\ref{fig:veg}d), but 4pm temperature is more correlated with vegetation than observations ($r= -0.48$, Fig.~\ref{fig:veg}f). This suggests that the model may overestimate the ability of vegetative land types to cool during the afternoon and slightly underestimate the vegetation/cooling relationship during the early morning. 

\subsubsection{Urban heat island intensity}
As discussed earlier, it is possible that the model may simply have a different diurnal cycle than observations which may cause statistics taken from a single hour (here, 6am or 4pm) to underestimate model performance. To address this, we examine the diurnal variability in temperature for each land type, shown in Fig.~\ref{fig:diurnal}c. At all sites, modeled temperature reaches its minimum at 6am and achieves maximum temperature at 1pm, cooling slightly until  7pm, at which point temperatures cool rapidly throughout the night. As previously seen in the histograms in Fig.~\ref{fig:hist}, 
temperature in high intensity land types is warmest throughout the day and rural forests are cooler throughout the day. Medium and low intensity land and open spaces are most similar to each other, becoming more similar to high intensity land during the day and and more similar to rural forests at night. 
This diurnal cycle contrasts with temperature observations in several ways: first, $T_{min}$ is too cold, second, $T_{max}$ is too early, and third, $T_{max}$ is too hot. That is, the model exaggerates the range of the diurnal cycle with respect to observations, and incorrectly reproduces the timing of the diurnal temperature cycle. Additionally, high intensity temperature differs too much from other land types at night, while other land types are too similar from the early evening to the afternoon. 
 
% This has implications for urban-rural temperature differences. The diurnal cycle of modeled $\Delta T$ is shown in Fig.~\ref{fig:diurnal}d, and reaches a maximum around 7pm and a minimum at 9am for high intensity land and 5am for other types of land. $\Delta T$ for high intensity land remains greater than 1$^\circ$C throughout the day, whereas $\Delta T$ in medium and low intensity land and open space is negative in the early morning, eventually growing to $1.95^\circ$C at 7pm. This temperature maximum lags that of high intensity land by one hour. This is in sharp contrast to observations, which are warmer in the morning and grow cooler into the evening and night. Thus, the model incorrectly produces the magnitude and timing of the diurnal $\Delta T$ cycle. % which may have implications on using the model to understand green infrastructure
 This has implications for urban-rural temperature differences. The diurnal cycle of modeled $\Delta T$ is shown in Fig.~\ref{fig:dt_sensitivity} for each rural site. 
 For the modeled BWI site (Fig.~\ref{fig:dt_sensitivity}f), $\Delta T$ reaches a maximum around 7pm and a minimum at 9am for high intensity land and 5am for other types of land. $\Delta T$ for high intensity land remains greater than 1$^\circ$C throughout the day, whereas $\Delta T$ in medium and low intensity land and open space is negative in the early morning, eventually growing to $1.95^\circ$C at 7pm. This temperature maximum lags that of high intensity land by one hour. This is in sharp contrast to observations, which are warmer in the morning and grow cooler into the evening and night. 
 At modeled sites 68 and 75 (Fig.~\ref{fig:dt_sensitivity}g,h), $\Delta T$ is small at night, reaching a minimum at 6am and increasing throughout the day until 8pm . The high intensity land has a higher $\Delta T$ than the less developed land types, all of which have smaller changes in $\Delta T$ throughout the day. 
Site 103 is the closest of the rural sites to the city  and has a minimum at 9am, and a maximum at 1am (Fig.~\ref{fig:dt_sensitivity}i). 
Rural forests have a negative $\Delta T$ throughout most of the day, indicating that site 103 is warmer than most of these sites. 
Finally, at site 130 (Fig.~\ref{fig:dt_sensitivity}j), $\Delta T$ varies little ($.8^\circ $C) throughout the day, with a minimum $\Delta T$ occuring at 10am and a maximum occuring at 9pm. 

% Maybe I need to stare at this figure longer, but to my eye all of the sites look pretty similar for the model. Yes, there are small differences, but the general picture is that UHI is high at night and into early morning, drops quickly in the morning, grows a little over the day, and then grows quickly after sunset.  This phrasing suggests that BWI is very different from the other sites.
At all sites,% except the BWI site, 
the minimum $\Delta T$ occurs in the early morning, between 5-9am, grows a little over the day, and grows quickly after sunset. % $\Delta T$ increases through the evening. 
For high intensity land type, $\Delta T$ varies throughout the day; however for other land types, this is not the case, and the variability is only around $1^\circ$C, lower than that for observations.  
The timing and magnitude of the diurnal $\Delta T$ cycle varies little when changing the rural reference station, in contrast to observations, which change significantly in timing and intensity depending on the rural reference. 
Modeled $\Delta T$ varies throughout the day in high intensity land, but other land types do not, further emphasizing that the model underestimates spatial variability.  
This shows that while the model produces a UHI at certain times of the day in certain land types, it does not reproduce the variability seen in observations.
%Furthermore, the model incorrectly produces the magnitude and timing of the diurnal $\Delta T$ cycle at the rural sites which are not located under a forest canopy.  % which may have implications on using the model to understand green infrastructure
%Having seen how sensitive observed $\Delta T$ is to the rural reference site, we now test how sensitive modeled $\Delta T$ is to rural site selection by comparing the model's urban temperature to temperature in the closest grid point to the BWI station (Fig.~\ref{fig:hist}f).Qualitatively, we see that high intensity land exhibits a similar diurnal curve to that in Fig.~\ref{fig:hist}d, with higher values at night and lower values during the day. The timing of this cycle is shifted and the values are lower, with the minimum $\Delta T$ of $0.1^\circ$C occuring at noon and the maximum of $2.1^\circ$C occurring at 10pm. The less developed land types show little significant difference throughout the day. This demonstrates that modelled $\Delta T$ is less sensitive to rural site selection in high intensity land. 

As the previous analysis uses only a single grid cell, it leaves open the possibility that model point selection may influence our results. Thus, we repeat our analysis using all points in a 40 square kilometer domain around Baltimore from our selected land types (developed high, medium, and low intensity land, open space, and deciduous forest). In Fig.~\ref{fig:diurnal_wd}a, we show temperature variability by hour for each of the 13 land types present in this domain. We see that developed land is hotter for most of the day, with the exception of water at night and wetlands during the early morning. %The spread between 
In Fig.~\ref{fig:diurnal_wd}b, we compare these diurnal cycles to the average diurnal cycle in a deciduous forest, representing 179 model points. With the exception of water at night and wetlands during the mid-morning, high intensity land types have the highest values of $\Delta T$ at all hours of the day. Similar to Fig.~\ref{fig:diurnal}, $\Delta T$ in high intensity land is around 2$^\circ$C at night, growing smaller until 9am, and then increasing throughout the day until 7pm. Medium and low intensity and open space show low $\Delta T$ at night, increasing at 6am throughout the day until 7pm, and then decreasing sharply thereafter. Again, this diurnal pattern is similar to that seen in Fig.~\ref{fig:diurnal} for each land type, demonstrating that modeled $\Delta T$ is less sensitive to rural site selection than observed $\Delta T$. 

\begin{figure}
\centering
\includegraphics[width=\textwidth]{rnw_chapter4/figures/whole_domain/diurnal_model.eps}
\caption{As in Fig. \ref{fig:diurnal}c,d but for the whole model domain. Number of points used to make each curve $N$ are reported in the legend.}% (a) Average summer temperature and (b) temperature difference by hour for for each landcover type.}% N indicates the number of model grid points included in each land type. }
\label{fig:diurnal_wd}
\end{figure}

 
 % note: add temperature labels on top and bottom of plots
\subsubsection{Surface energy budget}
% 4. The longterm flux balance should be $r_{net}=LW_{down}-LW_{up}+SW_{down}-SW_{up}=LH+HFLX-GRD$

\begin{figure}[h]
\centering
\includegraphics[width=\textwidth]{rnw_chapter4/figures/SEB.eps}
\caption{Average hourly surface energy flux by land type: a) net radiation $R_{net}$ (down minus up), b) sensible heat $HFX$, c) latent heat $LH$, and d) ground flux $GRDFLX$. Error bars represent spatial variability for time-mean hourly values.}
\label{fig:seb}
\end{figure}

We have seen that temperature varies throughout the urban heat island, and these differences are linked to surface features such as land type and vegetation in both the model and observations. 
In order to understand what causes the differences in temperature between each land type, we examine the surface energy budget in each land type (Fig.~\ref{fig:seb}). 
%A higher energy flux for more developed land types is consistent with warmer developed land and a pronounced UHI. 
 To understand the causes of this forcing, we break down the surface energy budget into component parts. First, we examine the net radiation fluxes in Fig.~\ref{fig:seb}a. 
% These fluxes differ little by land type, and we can attribute these differences to albedo and emissivity, which are assigned to each land type (see Table~\ref{tab:lcc}). 
These fluxes differ little by land type, indicating that differences in emissivity or in albedo (see Table~\ref{tab:lcc}), which would each have a direct impact on the radiation balance, do not drive differential heating between land cover classes.
For each land type, incoming shortwave radiation is the same (not shown), while reflected shortwave radiation is highest for less developed land types, which have higher albedo. 
 %Shortwave difference
% The differences in shortwave radiation (not shown) between land types are driven by albedo differences; in this model, albedo is a function of land type (Table~\ref{tab:lcc}). 
Developed land is less emissive but warmer, resulting in more outgoing longwave radiation in developed land. %Overall, this balances as more net radiation in rural areas, and similar net radiation in developed land types.  %causing a difference in net longwave radiation between land types. 
The sum of these differences are small--the net radiation is slightly higher in rural areas than in developed land types, but these differences are small (<$50$W/m$^2$) compared to the other fluxes. 

Sensible heat, by contrast, differs significantly by land type (Fig.~\ref{fig:seb}b).  
We see that sensible heat is highest for high intensity land and lowest for rural forests. The reverse is true for latent heating, which is highest for rural forests and lowest for urban areas  (Fig.~\ref{fig:seb}d). 
In the rural areas, 
greater soil moisture and the presence of vegetation lead to higher evapotranspiration.
The sum of the turbulent heat fluxes---sensible and latent heat---is larger in rural than urban areas, meaning that more energy enters the lower atmosphere as heat from these fluxes, resulting in more rural surface cooling. 
Next, we examine the ground flux (Fig.~\ref{fig:seb}d); this removes more heat from the surface in urban areas than in rural areas, partially offsetting the increased turbulent heat flux in rural areas. 
A key difference between ground heat flux and the turbulent fluxes (latent and sensible heat) is that the atmosphere can carry away energy lost as latent or sensible heating.
 Ground heat flux returns locally--if the ground absorbs more during the day it will return more at night or over longer time scales.
 %doesn't seem to be true... GRD flux But the sum of the turbulent heat fluxes---sensible and latent heat---is larger in rural than urban areas, meaning that more energy enters the lower atmosphere as heat from these fluxes, resulting in more rural surface cooling. 
In rural areas, slightly larger net radiation during the daytime is more than offset by latent heat losses; at night, sensible heat losses and radiative cooling cause rural areas to cool faster and little energy goes into ground storage. 
By contrast, urban surfaces experience more turbulent heat loss during the day than rural areas; the remaining heat is stored as ground storage, and returns at night as heat into the atmosphere,  resulting in warmer nighttime temperatures and a pronounced UHI. 
The difference influxes between land types remains small, however. 
For all fluxes, we see little variability between land type, and that fluxes are the same for open land as for medium and low intensity land. This suggests that land cover based parameterizations in the model fail to capture surface energy budget contrasts that are relevant to the UHI.
%Distinctions between temperature in each land class show that 

%Additionally, our observational results demonstrate that urban-rural differences in the model are too at night and too high during the day..... 
% notes: look at LW outgoing and see differences. this will correlate better with cooling differences. That is, should plot 1) LW out as the net, 2) HFX 3)LH 4) grd
% Also- try and plot urban-rural differences
%We see that $F$ is highest for  High intensity land, with medium, low and open space having the same net flux. Rural forests show the lowest net flux. Error bars show that for each land type, there is little spatial variability.  
%To diagnose why, we turn to the components of the surface energy budget (eq.~\ref{eq:seb}). The sensible heat flux (Fig.~\ref{fig:seb}b) is highest for high intensity land, and lowest for the rural forest. 
%In WRF, emissivity is fixed for each land cover class, though the vegetative fraction is not. Emissivity for high intensity land is $0.88$, lower than 
%\begin{table}
%\begin{tabular}{lclll}
%Parameter & name & 1 & 2 & 3 \\
%Urban fraction &  \textsc{FRC\_URB} & 0.5 & 0.9 & 0.95 \\
%Roof level &\textsc{ZR} &5.0&  7.5 & 10.0 \\
%Roof height std. deviation & \textsc{SIGMA\_ZED} & 1.0 &  3.0 &  4.0\\
%Roof width &  \textsc{ROOF\_WIDTH} & 8.3& 9.4 & 10.0 \\
%Road width & \textsc{ROAD\_WIDTH} &  8.3&  9.4&  10.0 
%\end{tabular} 
%\caption{WRF-Urban parameters which vary by urban land type 1-3, where 1 is lowest density and 3 is the highest density.}
%\label{tab:urban_model}
%\end{table}

As discussed in Methods, in urban areas, the model combines anthropogenic surface fluxes with fluxes from natural surfaces into a total urban surface flux $Q$:
$$ Q= F_{veg} \cdot Q_{veg} + F_{urban} \cdot Q_{urban}$$
where $F_{veg}$ and $F_{urban}$ are the fraction of vegetated and urban land respectively and $Q_{veg}$ and $Q_{urban}$ are the fluxes from the vegetated and urban land respectively. 
The urban fraction $F_{urban}$ for low, medium, and high intensity land are 0.5, 0.9, and 0.95 respectively, meaning that the urban model contributes very little to the total fluxes in low intensity land compared to medium and high intensity lands.
The model does not save out $Q_{veg}$ and $Q_{urban}$ explicitly, but it is reasonable to assume that $Q_{veg}$ and $Q_{urban}$ are of similar magnitude. 
Because $F_{veg}$ and $F_{urban}$ are so different in low versus medium land types, but not different between medium and high intensity land, the model parameter \textsc{FRC\_URB} explains why low intensity land differs from other land types. 

To explain why medium and high intensity land differs, we turn to the other parameters, including roof level, roof width, and road width. 
Each urban flux $F_{urban}$ is modeled as the total contributions from each anthropogenic surface, which includes roof, wall, and road: 
\[ F_{urban} = R\cdot F_{roof} + W\cdot F_{building} + RW\cdot F_{road} \]
where $R$ denotes the roof area fraction, $W$ denotes the building wall area fraction, $RW$ denotes the road area fraction.
Each of the fluxes (sensible heat, latent heat, and ground flux) is calculated in the same way for each land type, but the parameters $R, RW$, and $W$, which modify the urban surface area, differ by land type (Table~\ref{tab:urban_model}). 
These parameters govern the area of urban surfaces, including the roof height, width, and standard deviation, the road width, as well as their extent, namely, the fraction of urbanized land in each grid cell. 
With the exception of urban fraction, these parameters increase linearly with land use intensity, so that higher land uses have taller buildings, wider roofs, and larger roads. 
Other parameters are used in calculations for the energy fluxes, but do not change according to the land type. This includes the thermal conductivity, heat capacity, surface albedo, and surface emissivity, roughness lengths, and lower boundary temperatures. While these parameters are important in calculating the surface energy balance, that they do not change by urban land type indicates that they are not what causes the observed discrepancy in energy fluxes and thus temperature. 
Thus, we conclude that the parameters governing urban geometry cause the discrepancy of energy fluxes between urban land types. 
This suggests that future studies should modify the variables in Table~\ref{tab:urban_model} and conduct sensitivity tests in order to increase the spatial variability within the urban heat island.
 
\subsection{Model Discussion}
We performed a numerical simulation of Baltimore's urban heat island and evaluated modeled 2-meter temperature against observations. 
Using evaluation metrics commonly cited in the literature shows that the model error falls within the range reported by the literature, indicating that the model agrees well with observations. 
However, a closer examination of hourly temperature, the relationship between land use or vegetation and temperature, and the diurnal cycle of urban heat island intensity reveals that the model poorly captures several important aspects of Baltimore's urban heat island. 
%Our results show that error due to temporal variability is higher than that due to spatial variability, indicating that the model captures the spatial variability and extent of the urban heat island, though it poorly resolves the synoptic weather patterns. 
We also find %higher hourly error which indicate 
that the model exaggerates the diurnal temperature range, underestimating morning temperatures and overestimating afternoon temperatures. 
Additionally, the model significantly underestimates urban temperature's spatial variability, though it captures the patterns of spatial auto-correlation well as well as correctly reproduces the tendency for higher intensity land types to be warmer. 
Other characteristics are less well captured--we see significant differences from observations in that distributions of each land type do not overlap and show much less variability. 
Additionally, the model fails to capture the sensitivity of $\Delta T$ to rural station selection. 
The incorrect modeled diurnal temperature cycle causes the diurnal cycle of $\Delta T$ to be the opposite of observations for rural sites located in open areas. 
While we observe that urban-rural differences are largest in the morning and grow smaller during the day for open rural sites, the model suggests that at these sites, morning $\Delta T$ is low or even negative and that $\Delta T$ is largest at night. 
The model %also underestimates the distribution of vegetative fraction, particularly in high intensity land types. While it
correctly captures the negative relationship between large-scale vegetation (as observed by satellite) and temperature at 6am, it overestimates that same relationship at 4pm. This has possible policy implications-while the model suggests that greening policies can affect afternoon temperature, observations indicate that vegetation is only weakly linked with afternoon temperatures.
% 

%Modeled temperature is much less spatially variable than observed temperature. 
The model reproduces differences between land types but exaggerates these differences and underestimates variability within each land type. Analysis of the surface energy budget confirms that this homogeneity in each modeled variable extends to the surface energy fluxes.  
For example, the surface energy budget is the same for open land as for medium and low intensity land types, even though in reality those land types are different and should have different surface properties. While the surface energy fluxes provide insight into the causes of urban heat excess in the model, the lack of realism and variability between land types calls into question how useful this diagnosis is. 
We suggest that future studies should modify the variables in Table~\ref{tab:urban_model} and conduct sensitivity tests in order to increase the spatial variability within the urban heat island.

Additionally, our analysis suggests that the ground flux $G$ plays an important role in regulating UHI intensity. 
The radiation balance is usually invoked to explain UHI diurnal dynamics (\textit{eg}, \cite{oke82}), a finding reinforced by the difference in $\Delta T$ found in forests versus open fields.
Our results suggest that WRF may be compensating for a lack of a radiation effects with a large G effect, with the result that the modelled UHI becomes controlled by surface heat flux conditions rather than by aspects of urban form and materials that influence radiation.
Future numerical studies should investigate the role of ground fluxes in determining diurnal UHI dynamics. 

Studies that have used numerical models to develop process-based understanding of the UHI (\citep{zhang2011impact,li2013modeling,li2015contrasting,li2013synergistic}) should take into consideration that numerical models may over-simplify the variability present between urbanized land types. 
This suggests that modeling studies may benefit from using more stations to evaluate simulations as well as more detailed analysis. We suggest that studies unable to establish such extensive networks should endeavor to use evaluation stations in a range of land types, both for urban as well as rural land, to correct for possible sampling bias. 
Furthermore, our results demonstrate that rural site selection is a critical part of the urban-rural site comparison. 
This has important implications for both observational and modeling studies.
While it has been suggested that model bias may be neglected when computing urban-rural differences because the biases will cancel \citep{li2013synergistic}, our $\Delta T$ results show that this is not necessarily the case. 
We suggest that future UHI studies examine the validity of this assumption. 
% li and bou-zeid paper : hw

%zhang2011impact: upwind heating 
\section{Discussion}
Our analysis of Baltimore's urban heat island using observations and models demonstrates that challenges remain when investigating urban climate. 
While these challenges are generally acknowledged in model-based studies of urban climate; the findings of this study suggest that they can drive potentially erroneous conclusions regarding UHI process and mitigation potential. 
Conversely, challenges remain for monitoring temperature at urban scale. We find that instrument siting is critical to studying urban climate and making process-driven conclusions, as is designing network measurements in an appropriate way. 
This is difficult in an urban context, where securing permission for hundreds of instruments poses a significant challenge, and permission must take equal preference to scientific siting design concerns. 
In this study, shading errors affect our results and the assumption of a uniform error correction across space may not hold. 

In this study, we have examined several questions using micro-climate observations and models. Our results show that the answers to these questions may vary; similarly, our confidence in the answer to each question is variable and so we rate the confidence of our findings (Table \ref{tab:confidence}). We began by asking the question of 
how the UHI intensity varies with distance by looking at semi-variance in observations and models. For both techniques, we found similar results, and we rate our confidence in this finding as high. 
Next, we ask about the relationship between urban temperature and land type. We find that sensors capture this relationship well, but that the model struggles to represent variability in land types. 
Thus, we rate this finding as high confidence for observations and medium confidence for the model. 
Next, the relationship between green infrastructure and land type: we find lower observed temperatures linked with green infrastructure, so we rate this as high confidence. As the model  
Finally, we look at temporal variability in UHI intensity $\Delta T$. We find this to be highly variable for both urban and rural observations, but in the model, we find little variability. Thus, we rate this finding as low confidence for both findings. 

\begin{table}
\centering
\begin{tabular}{p{9cm} l l }
\toprule
Research question & Observations &  Model \\
\midrule
How does the UHI intensity  vary with distance? & high & high \\
What is the relationship between urban temperature and land type? & high & medium \\
What is the relationship between green infrastructure and land type? & high & N/A \\
When is $\Delta T$ highest? & low & low \\
\bottomrule
\end{tabular}
\caption{Confidence level of findings in this paper by methodology type.}
\label{tab:confidence}
\end{table}


\section{Conclusions}\label{sec:conclusions}
Observations from an urban network of thermometers reveal several key characteristics of Baltimore's urban heat island. We show that spatial variability is around $1^\circ$C and attributed to land type, with higher intensity land being hotter. 
Furthermore, temperatures are negatively correlated with the presence of both large scale and local vegetation. This relationship is time dependent for both scales---at 4pm, the correlation with large-scale vegetation is insignificant, whereas at 6am, the correlation is strong, and for small-scale vegetation, different green infrastructure types cool at different times of day.  
This demonstrates that green infrastructure such as parks, street trees, or urban forests can cool the urban heat island, though the magnitude of the effect depends on the time of day: urban forests cool the most during the day, while parks offer the most cooling at night. 
We also find a relationship between temperature and distance to the nearest park, though only at night.
 Urban-rural differences $\Delta T$ are highest at 6am and lowest at 4pm for open rural sites, but for forested sites, the opposite is true. Thus $\Delta T$ varies by land type as well as with $T_r$.

 We use these results to evaluate a numerical simulation of Baltimore in WRF, and show that while evaluation statistics such as the RMSE (4.6 for temporal error, 0.93 for spatial error) are in line with RMSEs of $0.6-4.0^\circ$C reported in the literature  \citep{kim2013evaluation}, the model fails to capture many of the observed UHI characteristics. In particular, the model exaggerates the diurnal range of temperatures, incorrectly estimates the timing and intensity of $\Delta T$ for most rural sites, and over-estimates the daytime relationship between vegetation and temperature. 
 Additionally, the model underestimates variability within each land type;%, suggesting that the city is more homogeneous than reality; 
 we link this finding to the surface energy flux, which we find to be the same for differing land types. This suggests that summary statistics may not accurately diagnose the skill of numerical simulations at simulating key aspects of the urban heat island. 

While our numerical model has a mean error with respect to urban heat island observations that is similar to other studies, particularly for spatial error, it does not capture either the timing or intensity of the diurnal cycle of temperature and urban-rural differences, and overestimates the relationship between daytime temperature and vegetation. This suggests that care must be taken when using numerical models to understand urban heat island processes and to prescribe urban heat island mitigation policies. 
%Indeed, our results call into question a number of claims made by modeling studies of Baltimore, as they demonstrate that models fail to capture the defining aspects of Baltimore's UHI. 
In addition to evaluating the overall model error and bias, it is important that future studies evaluate the ability of any urban heat island model to describe key observed features of that urban heat island. 

%Additionally, our microclimate network shows that....

Our study focuses on one numerical model. While we find that this model reproduces UHI features found by using other urban canopy models, our results show the limitations of understanding urban heat island process through numerical models and demonstrates that it is not sufficient to assume that a low bias and small error indicate that the model accurately represents the urban heat island.  
While some studies report UHI features such as the diurnal UHI cycle, many studies do not. 
This study demonstrates that model error is different in urban and rural areas, which means that assumptions of errors canceling each other out when comparing urban and rural areas may be incorrect. 
%Scientists must take care when assuming that model error is spatially or temporally constant; 
We suggest that future studies either assess this error or, if not possible, resist comparing heterogeneous sites. This is critically important when such studies make policy recommendations, which many UHI mitigation studies do: for example, studies investigating the role of green infrastructure in urban heat island mitigation should first validate the ability of their model to capture the vegetation-temperature relationship. 
We find differences between the model and observations that would affect both scientific understanding of UHI process as well as policy recommendations. 
Thus, care must be taken when using urban heat island models to understand process, predict future change, and recommend interventions when no detailed evaluation has been performed. We call on future studies to expand their evaluation metrics for urban heat island studies in order to improve scientific understanding and support disaster management professionals, healthcare workers, and policy makers in making data-driven decisions. 