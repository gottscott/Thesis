\chapter{Conclusions}
\label{chap:conclusion}
% big opening sentence
\subsection{Summary}
This thesis examined urban temperature variability over several spatial and temporal scales using observations, satellite data, and models. 
We have looked at temperature variability on multiple spatial and temporal scales, examining both variability within cities and continental scale variations as well as longer term trends in an attempt to characterize and thus better understand the phenomena known as the urban heat island.
 Our results show that drawing conclusions on urban phenomena requires a breadth of techniques and benefits from examination over several timescales.  
 
% conclusions from chap 1 
In Chapter 1, we examined summertime minimum daily temperature in Baltimore, Md. using 135 low-cost air temperature sensors. We saw that much of the spatial variability is small and that this variability is explained by surface properties, namely, the presence or absence of vegetation, and not well explained by meteorological variability. The time-averaged minimum daily temperatures have a spatial standard deviation ($0.9$) that is much smaller than the same measure for satellite-derived land surface temperature ($2.07$), and the sensor-measured temperatures agree well with the NOAA-NWS weather station in downtown Baltimore, with an arithmetic mean difference for all measurements in time and space of $0.00^{\circ}C$. The presence or absence of vegetation affected temperature more than other meteorological and surface properties examined, and time-averaged air temperatures in green spaces are found to be cooler than impervious spaces by about $1^{\circ} C$. Additionally, only temperatures measured inside green spaces correlate significantly with surface properties, in particular tree cover and elevation, whereas temperatures measured over impervious surfaces do not. 

In Chapter \ref{chap:nairobi}, we presented a similar study performed in a very different environment. Looking at temperatures in Nairobi, Kenya, we showed that 
% Nairobi conclusions 
during the hottest summer on record in Nairobi, temperatures measured within three of Nairobi's informal settlement neighborhoods--- Kibera, Mathare, and Mukuru--- regularly exceed temperatures at the central, non-slum monitoring station by several degrees or more. Unlike the previous chapter, we examined daytime as well as nighttime temperatures and saw that the observed temperature differences persist throughout the day and night.  The spatial patterns observed in temperature are consistent over the measurement period---that is to say, hot stations remain hotter, and cool stations remain cooler. This spatial variability increases during periods of extreme heat, though the mean differences with the station tend to be smaller as it warms. This is particularly true for maximum daily temperature, though the station remains hotter than the spatially averaged temperature for the entire measurement period. We can connect much (66\%) of the site-to-site variability in mean temperature to surface properties. In particular, the presence of vegetation (measured remotely and \textit{in situ}) is a significant predictor of cooler mean temperatures.

Next, we turned to a larger scale analysis of urban-rural temperature differences. In Chapter \ref{chap:hw}, we compared daily maximum and minimum urban and rural temperatures using station observations from 54 US cities for 2000-2015. This work shows that in most cities, the intensity of the UHI, $\Delta T$, diminishes with warmer temperatures.  We further showed that the result holds over temporal scales---daily over the entire summer, during extreme heat---as well as across climate zones. We found that for daily variability $T_u$ increases on average by only 0.88 $^\circ$C for every degree increase in $T_r$, while the average decrease of $T_u$ during extreme heat is 0.93 $^\circ$C. %for every degree increase in $T_r$.  
We also showed also that the results held for the longer period of 1985-2015, a time period when more inconsistencies in the observational record could have have influenced results.
Our focus on pairs of stations rather than urban networks allows us to look at interannual trends, unlike previous chapters. Here we find that urban trends exhibit a one-to-one relationship with average temperature trends, meaning the UHI has not increased with climate change, but do note that rural trends decreased slightly with increasing average trends. We relate the decrease in $\Delta T$ with larger $T_r$ to large-scale or synoptic weather conditions, and find that the lowest $\Delta T$ nights occur during moist conditions. Further, we find that across all time and space scales, our results are driven by changes in rural temperatures rather than urban temperatures, which appear less sensitive to both heat and weather type. 

%The work presented in Chapter~\ref{chap:hw} addresses how urban heating would decrease under warmer conditions, especially when the scientific literature relying on theory and modeling has suggested the opposite.  
%A fuller examination of the findings in in Chapter~\ref{chap:hw} would require a numerical model in order to thoroughly examine all the variables and parameters of interest. 
%However, before undertaking a process based study in a numerical model, that model should first be evaluated for its suitability to address the problem of interest. 
In Chapter~\ref{chap:bmore2}, we returned to the city scale and re-examined Baltimore, expanding beyond the neighborhood scale examined in Chapter~\ref{chap:bmore} with a temperature network of unprecedented size and density. This network was used to
%Thus, we turn to a model evaluation using city scale data in Chapter \ref{chap:bmore2}.
%We return to Baltimore and 
quantify nighttime and daytime urban heat island characteristics. Using hourly temperatures for both daytime and nighttime, we showed that 
spatial variability is around $1^\circ$C throughout the City of Baltimore and attributed to land type, with higher intensity land being hotter. 
We link cooler temperatures both with the presence of large scale and local vegetation; this relationship is greater at 6am than at 4pm, when the correlation is low. The diurnal cycle shows that urban-rural differences $\Delta T$ are highest at 6am and lowest at 4pm, but vary by land type as well as the reference used. We use these results to evaluate a numerical simulation of Baltimore in WRF, and show that while evaluation statistics such as the RMSE are in line with other studies reported in the literature, the model fails to capture many of the observed characteristics. In particular, the model incorrectly diagnoses the diurnal cycle of temperature and $\Delta T$ and over-estimates the daytime relationship between vegetation and temperature. 

\subsection{Discussion}
% chapter 1 discussion 
The work presented in this thesis has a number of ramifications for both scientific and policy-relevant inquiries.
First, one outstanding question of interest to urban heat island researchers is how to relate land surface temperatures to air temperatures. Land surface temperature data is globally available from satellite imagery, unlike air temperature data, but is difficult to relate to thermal comfort and health indices which are used by decision makers and health practitioners. 
Our work suggests that thermal satellite imagery can exaggerate the variability of air temperatures and care must be used when using thermal imagery in place of \textit{in situ} air temperature measurements in order to diagnose urban heating. 
% Note: is this still true when you consider daytime differences?
In Baltimore, we found that the mean differences with the downtown weather station are not statistically significant even where satellites indicate land temperatures are the hottest. This findings support the use of the central weather station within Baltimore to assess average thermal conditions even in thermally-identified hot spots. 
However, our Nairobi study shows that this is not universally the case- our measurements suggest that Nairobi's Dagoretti weather station significantly underestimates the heat exposure that is experienced by residents of informal settlements. 
Furthermore, those differences can be concerning for health reasons, as 
the temperatures we measured within Nairobi's neighborhoods are within the range of temperatures that have previously been associated with negative health outcomes for children and elderly populations \citep{egondi2012}. In light of findings in later chapters, it is probable that we can explain this discrepancy to the local land types: in Nairobi, the central weather station is located in a different land type from the neighborhoods studied, whereas in Baltimore, both the station and neighborhoods are dominated by impervious surfaces and low buildings. 
Thus, we recommend future studies assess this possible discrepancy before selecting a weather station. 

Next, this thesis concerns how urban temperature changes during heatwaves. 
Despite reports in the literature that heatwaves exacerbate the urban heat island, we show in Chapter~\ref{chap:hw} that this is not universal, and that in many cases, heatwaves diminish the intensity of the urban heat island. 
Our results differ from previous studies because we examine more cities, longer time periods, and more heat events. Conclusions of an increasing UHI during warmer temperatures were based on studies of single cities and single events. For example, the previous study of Baltimore \citep{li2013synergistic} only examined one heat event; we examine this event and others in addition to 15 years of daily data and find that the event in \cite{li2013synergistic} is atypical. We do find some individual cities for which UHI increases with warmer temperatures, however, in most cases, we find this to be related to the land type and thus microclimate of the weather station, similar to \cite{zhou2010atlanta}. 
We also see some evidence that this is related to station siting- we note that the effect is more pronounced for cities where urban stations are located in higher intensity land types. As we saw in Chapter ~\ref{chap:bmore2} that rural station selection can change the timing and intensity of urban-rural temperature differences, future studies should characterize this effect using hourly data from a variety of locations. 

Additionally, the decrease in UHI intensity with warmer conditions has potentially important implications for accounting for urbanization in long term climate records, where it is often assumed that absent significant changes in urban extent, the urban and rural areas warm at the same rate \citep{hausfather2013quantifying,stone2012managing}.  While much of the literature has focused on accounting for and investigating potential urban bias, our results suggest that equal importance must be place on rural stations. 
Perhaps more importantly, it has implications for changes in the UHI as climate continues to warm as well as for economic projections of climate change impacts for cities \citep{estrada2017global}. 
%The summer median $\Delta T$ is 1.75 $^\circ$C and the interannual sensitivity of $T_u$ to changes in $T_r$ is ~$0.3 ^\circ $C/$^\circ $C, 
That warming areas did not experience any increase in urban heating indicates that the nighttime urban heat island has not been exacerbated by climate change.  Absent consideration of changes in synoptic weather patterns, urbanization, or urban extent, our results suggest that the urban heat island as defined by available weather stations may remain constant or possibly decline as background climate warms. Indeed, there is already a tendency for $T_r > T_u$ on the hottest nights for many cities (e.g., Fig. ~\ref{hw}d). Few studies 

Though these results suggest that climate change may diminish urban-rural temperature differences, 
we emphasize that this does not mean that global warming will not affect cities. 
Rather, it suggests that surrounding rural areas may warm faster than urban areas absent changes in urbanization and urban extent. As demographers continue to forecast urban population growth well into this century, these last two assumptions are unlikely to hold true, particularly in developing environs where the most  climate-vulnerable populations live. 
Furthermore, the moist weather types which we associate with low UHI nights, in particular the moist tropical weather type, are associated with elevated risks for mortality and morbidity  \citep{sheridan2004progress}, meaning that lower UHIs will not necessarily translate into lower health risks. 
This is important for heat mitigation efforts, economic projections, and climate resiliency plans to take into account, as our results suggest that rural and suburban heat mitigation efforts may be more important than previously thought. Health analyses are concerned with physiologically relevant temperature thresholds, and our results indicate that assumptions of a constant or increasing urban heat may mischaracterize those risks. 
More work must be done to understand how urban heat island intensity evolves on interannual timescales. 

Finally, this study suggests the policy measures that could be implemented to better mitigate urban heating. While previous studies have demonstrated the feasibility of green infrastructure to mitigate urban heating, our results clarify when different greening strategies may be most effective.
In Baltimore, our results demonstrate that green infrastructure such as parks, street trees, or urban forests can cool the urban heat island, and that different green types cool at different times: urban forests cool the most during the day, while parks offer the most cooling at night. 
Our results suggest that some of Baltimore and Nairobi's urban heating may be mitigated through improved urban design and increased greenery, though more data is needed to better assess this in Nairobi, where a more limited set of urban greenery was evaluated. 
This work raises a number of questions. 
While we suggest that radiative cooling explains why green spaces are cooler at night, more data is needed to confirm this. Furthermore, more detailed energy budget data is needed to diagnose how and when green infrastructure cools. Finally, this work raises the question of how these findings will differ when examining cities in differing climate zones. 
%Another question is whether our findings will hold when examined on a city-wide basis.  More work is ongoing to determine this. Hopefully this will help answer questions about how densely temperature must be monitored to capture the sub-neighborhood variability of interest to our partners in public health and urban planning. This study examined areas that were largely homogeneous in terms of the built environment, but for a geographically expanded study, pairing data with a standard measure of urbanization or classification such as a brightness index or the local climate zone classification could help comparisons with ongoing work in other cities.
%In all cases, we stress that these policy measures be more effectively assessed using field experiments, observations from cities, or appropriate models before recommending implementation. 

However, care must be taken when selecting a model to answer such questions. 
While our WRF simulations have a low overall error compared with urban heat island observations, it does not capture either the timing or intensity of the diurnal cycle of temperature and urban-rural differences, and overestimates the relationship between daytime temperature and vegetation. This suggests that care must be taken when using numerical models to understand urban heat island processes and to prescribe urban heat island mitigation policies. In addition to evaluating the overall model error and bias, it is important to evaluate the ability of any urban heat island model to describe key observed features of that urban heat island. 

\subsection{Future Work}
This thesis also leaves several questions unanswered. 
Our Nairobi study demonstrates the need for more studies in tropical cities, Southern Hemisphere studies, and cities where the urban typography differs significantly from the North American, European, and more recently East Asian cities which dominate the literature. An additional emphasis deserves to be placed on low and middle income cities, where rapid urbanization and global warming may outstrip available resources for treating and preventing heat exposure. 
Our work leaves several questions outstanding for understanding  heatwaves, the heat island, and their interaction in Nairobi and similar cities. Future studies should consider a broader range of neighborhoods to understand how heat and heat exposure may affect all residents. Finally, we note that we took measurements in an extraordinarily hot year. As temperatures rise and hot summers like 2015 become more common, a fuller understanding of how the urban heat island interacts with heatwaves will become more important. 

Future, researchers must continue to thoroughly evaluate the ability of numerical models to answer the specific questions of the urban heat island researcher.  
While our study focuses on a single numerical model, our results show the limitations of understanding urban heat island process through numerical models and demonstrates that it is not sufficient to assume that a low bias and error indicate that the model accurately represents the urban heat island.  While some studies report UHI features such as the diurnal UHI cycle, many studies do not. Our work shows that this is something future studies must take into account. 

Furthermore, our work demonstrates that model error differs by land type. 
This means that assumptions of errors canceling each other out when comparing urban and rural areas may be incorrect. Similarly, few studies investigating the role of green infrastructure in urban heat island mitigation first validated the ability of their model to capture the vegetation-temperature relationship. We find differences between the model and observations that would affect both scientific understanding of UHI process as well as policy recommendations. 
Thus, care must be taken when using urban heat island models to understand process, predict future change, and recommend interventions before a detailed evaluation has been performed. We call on future studies to expand their evaluation metrics for urban heat island studies in order to improve scientific understanding and support decision makers in making data-driven decisions. 

Overall, our work shows that much is left to be done in order to understand the urban heat island and answer the science and policy questions necessary to usher cities safely into a warming, urbanizing world. 
While our results are limited to a small subset of cities, they show how targeted fieldwork can complement traditional datasets and modeling techniques.
As urban heat island science moves towards studying traditionally under-studied cities in Asia and beyond, it is important to continue to insist on ground-truthing of satellite information and evaluation of numerica models using local, \textit{in situ} data in order to best contextualize the health and policy ramifications of scientific measurements. By combining a variety of observational and modeling techniques, scientists can enhance and contribute to a process-based understanding of urban heating. 
Such an understanding may aid in the development of more targeted interventions to reduce the effects of heat exposure now and in a warmer climate.
% last chapter
